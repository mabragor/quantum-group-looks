\documentclass{article}

%\textheight 25cm \textwidth 17cm \voffset= - 1.2in \hoffset= - 1.0in
\usepackage{amsmath}
\usepackage{amsfonts}
\usepackage{amsthm}
\usepackage{amsopn}
\usepackage{verbatim}
\newcommand{\hp}{\frac{\hbar}{i} \partial_x}
\newcommand{\lb}{\left (}
\newcommand{\rb}{\right )}
\newcommand{\lsb}{\left [}
\newcommand{\rsb}{\right ]}
\newcommand{\lab}{\left \langle}
\newcommand{\rab}{\right \rangle}

\newcommand{\eq}[1]{\begin{equation} #1 \end{equation}}
\newcommand{\eqna}[1]{\begin{eqnarray} #1 \end{eqnarray}}
\newcommand{\be}{\begin{eqnarray}}
\newcommand{\ee}{\end{eqnarray}}
\newcommand{\bc}{\begin{comment}}
\newcommand{\ec}{\end{comment}}

\newcommand{\Lambdan}{\Lambda^{2(N-M)}}
\DeclareMathOperator{\Or}{or}
\DeclareMathOperator{\tr}{tr}
\DeclareMathOperator{\If}{if}
\DeclareMathOperator{\And1}{and}
\DeclareMathOperator{\that}{that}
\DeclareMathOperator{\is}{is}
\newcommand{\p}{\partial}
\newcommand{\mF}{\mathcal{F}}
\newcommand{\e}{\epsilon}
\newcommand{\wk}{\widetilde{k}}
\newcommand{\ok}{\overline{k}}
\newcommand{\D}{\mathcal{D}}
\newcommand{\cmO}{\hat{\mathcal{O}}}
\newcommand{\wLambda}{\widetilde{\Lambda}}
\newcommand {\?}{\textit{???}}
\newcommand{\wZ}{\widetilde{Z}}
\newcommand{\wF}{\widetilde{F}}
\newcommand{\wG}{\widetilde{G}}
\newcommand{\me}[0]{\mathcal{E}}
\newcommand{\hT}[0]{\hat{T}}
\newcommand{\Bp}[1]{B^{'}\lb #1 \rb}

\newcommand{\YAMLSectionize}[1]{}

\def\dg{\Delta (g)}
\def\gog{g \otimes g}
\def\classlim{q \rightarrow 1}
%\newcommand{\ll}[1]{\left ( #1 \rb}
\newcommand{\Echi}[1]{\me_{1/q} \lb \chi_#1 T_{-#1} \rb}
\newcommand{\Epsi}[1]{\me_{q} \lb\psi_#1 T_{+#1}\rb}
\newcommand{\Cartan}[1]{\lsb q^{2\phi_i H_i} \rsb}
\newcommand{\EchiFirst}[1]{\me_{1/q} \lb\chi_#1 T_{-#1} \otimes I \rb}
\newcommand{\EchiSecond}[1]{\me_{1/q} \lb\chi_#1 I \otimes T_{-#1} \rb}
\newcommand{\EpsiFirst}[1]{\me_{q} \lb\psi_#1 T_{+#1} \otimes I \rb}
\newcommand{\EpsiSecond}[1]{\me_{q} \lb\psi_#1 I \otimes T_{+#1} \rb}
\newcommand{\CartanFirst}[1]{\lsb q^{2\phi_#1 H_#1} \otimes I \rsb}
\newcommand{\CartanSecond}[1]{\lsb I \otimes q^{2\phi_#1 H_#1}  \rsb}
\newcommand{\CartanCoprod}[1]{\lsb q^{2\phi_#1 H_#1} \otimes q^{2\phi_#1 H_#1}  \rsb}
\newcommand{\EchiPreCoprod}[1]{\me_{1/q} \lb\chi_#1 \Delta \lb T_{-#1} \rb \rb}
\newcommand{\EpsiPreCoprod}[1]{\me_{q} \lb\psi_#1 \Delta \lb T_{+#1} \rb \rb}
\newcommand{\EchiCoprod}[1]{\me_{1/q} \lb\chi_#1 \lb q^{2H_i} \otimes T_{-#1} + T_{-#1} \otimes I\rb \rb}
\newcommand{\EpsiCoprod}[1]{\me_{q} \lb\psi_#1 \lb I \otimes T_{+#1} + T_{+#1} \otimes q^{-2H_i}\rb \rb}
\newcommand{\EchiTwisted}[1]{\me_{1/q} \lb\chi_#1 q^{2H_i} \otimes T_{-#1} \rb}
\newcommand{\EpsiTwisted}[1]{\me_{q} \lb\psi_#1 T_{+#1} \otimes q^{-2H_i} \rb}


\def\Chi{\mathcal{X}}


%% \newcommand{\EChi}[0]{\me_{1/q} \lb \chi_#1 T_{-#1} \rb}
%% \newcommand{\EPsi}[0]{\me_{q} \lb\psi_#1 T_{+#1}\rb}
\newcommand{\EChiL}[0]{\Chi^L}
\newcommand{\EChiR}[0]{\Chi^R}
\newcommand{\EChiLT}[0]{\Chi^{Lt}}
\newcommand{\EChiRT}[0]{\Chi^{Rt}}
\newcommand{\EChiD}[0]{\Chi^\Delta}
\newcommand{\EPsiL}[0]{\Psi^L}
\newcommand{\EPsiR}[0]{\Psi^R}
\newcommand{\EPsiLT}[0]{\Psi^{Lt}}
\newcommand{\EPsiRT}[0]{\Psi^{Rt}}
\newcommand{\EPsiD}[0]{\Psi^\Delta}
\newcommand{\QPhiL}[0]{\Phi^L}
\newcommand{\QPhiR}[0]{\Phi^R}
\newcommand{\QPhiD}[0]{\Phi^\Delta}

\newcommand{\EPsii}[1]{\Psi_#1}
\newcommand{\EChii}[1]{\Chi_#1}
\newcommand{\QPhii}[1]{\Phi_#1}
\newcommand{\EChiLi}[1]{\Chi_#1^L}
\newcommand{\EChiRi}[1]{\Chi_#1^R}
\newcommand{\EChiLTi}[1]{\Chi_#1^{Lt}}
\newcommand{\EChiRTi}[1]{\Chi_#1^{Rt}}
\newcommand{\EChiDi}[1]{\Chi_#1^\Delta}
\newcommand{\EPsiLi}[1]{\Psi_#1^L}
\newcommand{\EPsiRi}[1]{\Psi_#1^R}
\newcommand{\EPsiLTi}[1]{\Psi_#1^{Lt}}
\newcommand{\EPsiRTi}[1]{\Psi_#1^{Rt}}
\newcommand{\EPsiDi}[1]{\Psi_#1^\Delta}
\newcommand{\QPhiLi}[1]{\Phi_#1^L}
\newcommand{\QPhiRi}[1]{\Phi_#1^R}
\newcommand{\QPhiDi}[1]{\Phi_#1^\Delta}
\newcommand{\EBlockLi}[1]{\EPsiLi{#1} \QPhiLi{#1} \EChiLi{#1}}
\newcommand{\EBlockRi}[1]{\EPsiRi{#1} \QPhiRi{#1} \EChiRi{#1}}
\newcommand{\ComultBlock}[1]{\EPsiLi{#1}\EPsiRTi{#1}\QPhiLi{#1}\QPhiRi{#1}\EChiLTi{#1}\EChiRi{#1}}

\newcommand{\matd}[4]{\lb \begin{array}{cc}
#1 & #2 \\ #3 & #4
\end{array} \rb}

\newcommand{\comul}[1]{\Delta \lb #1 \rb}

\newcommand{\delabel}[1]{(\ref{#1})}

\newcommand{\Hone}[1]{\lb \begin{array}{ccc}
#1^{2/3} & 0 & 0 \\ 0 & #1^{-1/3} & 0 \\ 0 & 0 & #1^{-1/3}
\end{array} \rb}
\newcommand{\Honed}[1]{\lb \begin{array}{ccc}
#1^{1/2} & 0 \\ 0 & #1^{-1/2}
\end{array} \rb}
\newcommand{\Eone}[1]{\lb \begin{array}{ccc}
1 & #1 & 0 \\ 0 & 1 & 0 \\ 0 & 0 & 1
\end{array} \rb}
\newcommand{\Eoned}[1]{\lb \begin{array}{cc}
1 & #1 \\ 0 & 1
\end{array} \rb}
\newcommand{\Fone}[1]{\lb \begin{array}{ccc}
1 & 0 & 0 \\ #1 & 1 & 0 \\ 0 & 0 & 1
\end{array} \rb}
\newcommand{\Foned}[1]{\lb \begin{array}{cc}
1 & 0 \\ #1 & 1
\end{array} \rb}


\newcommand{\Htwo}[1]{\lb \begin{array}{ccc}
#1^{1/3} & 0 & 0 \\ 0 & #1^{1/3} & 0 \\ 0 & 0 & #1^{-2/3}
\end{array} \rb}
\newcommand{\Etwo}[1]{\lb \begin{array}{ccc}
1 & 0 & 0 \\ 0 & 1 & #1 \\ 0 & 0 & 1
\end{array} \rb}
\newcommand{\Ftwo}[1]{\lb \begin{array}{ccc}
1 & 0 & 0 \\ 0 & 1 & 0 \\ 0 & #1 & 1
\end{array} \rb}


%%%%%%%%%%%%%%%%%%%%%%%%%%%%%%%%%%%%%%%%%%%%%%%%%%%%%%%%%%%%%%%%%%%%%%%%
%%%%%%%%%               SPACE FILLING SETTINGS               %%%%%%%%%%%
%%%%%%%%%%%%%%%%%%%%%%%%%%%%%%%%%%%%%%%%%%%%%%%%%%%%%%%%%%%%%%%%%%%%%%%%
\textheight 24.5cm
\textwidth 17cm
%\voffset=-1.2in
\voffset=-1.4in
%\voffset= - 1.85in
\hoffset= - 1.0in         % switch off for draft style
%%%%%%%%%%%%%%%%%%%%%%%%%%%%%%%%%%%%%%%%%%%%%%%%%%%%%%%%%%%%%%%%%%%%%%%%

\title{{\bf The cluster variety face of quantum groups} \vspace{.2cm}}
\author{{\bf A.Popolitov}\thanks{{\small
{\it ITEP, Moscow, Russia}}; popolit@itep.ru}}

\begin{document}
 \maketitle

\vspace{-5.0cm}

\begin{center}
\hfill ITEP/TH-\?\\
\end{center}

\vspace{3.5cm}

\centerline{ABSTRACT}

\bigskip

{\footnotesize
Using the well-known free-field formalism for quantum groups \cite{MV1}, we demonstrate
in case of $A(n)_q$, that quantum group is naturally also a cluster variety \cite{FG1}.
Widely used \cite{Hik1} formulae for mutations \cite{FG1},\cite{FG2} are
direct consequence of independency of group element on the order of simple roots.
Usual formulae \cite{Mars1} for $2 n$ Poisson leaf emerge in classical limit,
if all but few ($2n$) coordinates vanish.
{\it do not know, whether I should say here, that formulae are in fact a little
bit different}
}

%\tableofcontents

\YAMLSectionize{
# The sketch and logic of the presentation
- Introduction :
  - first a blah-blah-blah on how quantum groups are great and important objects,
    and cluster varieties - even more important
  - convenient MV parametrization of group element and commutational relations
  - convenient MFG parametrization of group element
  - mutations
- Body :
  - Details :
    commutational relations for dual algebra from comultiplication :
      - brief overview of original Morozov-Vinet
      - derivation for sl(2)
      - derivation for sl(3)
      - derivation for sl(n)
    connection between multicartan and Gauss parametrization
      - sl(2)
      - sl(n)
    commutational relations for multicartan parametrization
      - sl(2)
      - sl(n)
    mutations
      - sl(2) fundamental
      - sl(n) fundamental reduces to sl(2)
      - sl(2) first symmetrical
- Conclusion
}

\bigskip

\bigskip

%% \section{Introduction}

%% Quantum groups were studied by Leningrad school and particularly Drinfeld.
%% They arise as symmetries of non-commutative spaces, though their applications
%% are not limited by this.
%% Cluster varieties, on the other hand, originate from cluster algebras, which
%% appeared in works of Zelevinski et al, and were developed by Goncharov and Fock \?

%% In \? the explicit construction was developed to embed cluster variety into classical Lie group.
%% The claim was made (though not illustrated with formulas), that this can also be done in the quantum case.
%% The drawback (or the advantage) of this approach, however, is, that cluster varieties and
%% Lie groups still seem to be quite different objects, though an embeddings exist, that
%% respect the structures of these objects.

%% In this paper we demonstrate, for the case of quantum $SL(n)$ group, that quantum group
%% naturally is a (quantum) cluster variety, with certain intersection matrix \?.
%%  Namely, starting from certain version of ansatz for
%% quantum group element (long known as free-field formalism \?) and comultiplication rules.
%% We derive commutational relations for the algebra of functions on quantum group (the so-called {\it dual algebra}),
%% which turn out to be of the form of relations defined in \? and formulas for gluing maps between
%% parametrisations, related to different equivalent choices of ordering of simple roots
%% turn out to be of the form of (quantum) mutations, also defined in \?.

%% This requirement of correspondence between quantum group setting and cluster variety setting
%% allows us to fix both constructions a little bit.
%% On one hand, we find, that certain subfamily of ansaetze, suggested in \?,
%% has most simple (Darboux-like) commutational relations.
%% On the other hand, only particular ``words'' of roots in the construction from \?
%% allow to parametrize whole quantum group.
%% Also we find, that lower-indexed, not upped-indexed basis in Cartan subalgebra is involved in the construction.

%% This last fact indicates, either, that some other definition of quantum group is possible
%% (involving the other basis in the rules for comultiplications),
%% or that classical construction, presented in \? is somehow isomorphic to to the one we find here.

%% This, however, is out of scope of this paper and is left for further investigation.

\section{Introduction/Cheatsheet}
Quantum groups are very cool and everywhere-important objects.

So do cluster varieties.

Let's briefly summarize the progress made in this paper.
If this section looks somewhat overgeneric or sketchy to you, you can safely skip to the next
section, where ideas are elaborated in the simplest case of $SL(2)_q$.

First, we take the quantum group element $g$ to be of the form
of the word of elementary {\it building blocks} $B(i)$, each related to some positive-negative pair
of simple roots $\alpha_{\pm \lsb i \rsb}$
\be
g = B \lb i_1 \rb B \lb i_2 \rb \dots B \lb i_{n^2-1} \rb,
\ee
where building block $B(i)$ can be expressed in one of the two equivalent forms
\be
B(i) = \me_q \lb \psi_i q^{H_{[i]}} \hT_{+[i]} \rb q^{\phi_i H_{[i]}} \me_{1/q} \lb \chi_i \hT_{-[i]} q^{-H_{[i]}} \rb
\equiv
w_i^{H_{[i]}} \me_q \lb q^{H_{[i]}} \hT_{+[i]} \rb x_i^{H_{[i]}} \me_{1/q} \lb \hT_{-[i]} q^{-H_{[i]}} \rb y_i^{H_{[i]}},
\ee
where $\me_q(x)$ is $q$-exponential, and $H_{[i]}$, $T_{+[i]}$ and $T_{-[i]}$ are Chevalley generators of the algebra.
Square brackets denote the \? map from $n^2-1-n$ parameters to only {\it simple} ones.

Equivalence between two building block parametrisations is given by
\be
\psi_i = w_i,\ \ \chi_i = y_i,\ \ q^{\phi_i} = w_i x_i y_i = y_i x_i w_i
\ee

Quantum group defining equation $\dg = \gog$ then implies the Darboux-like commutational
relations on the quantum group parameters (also called elements of the dual algebra).
\be
& \psi_i \chi_j = \chi_j \psi_i, \ \ q^{\phi_i} \psi_j = q^{2 \delta_{ij}} \psi_j q^{\phi_i}, \ \
q^{\phi_i} \chi_j = q^{2 \delta_{ij}} \chi_j q^{\phi_i} & \\ &
y_i w_j = w_j y_i, \ \ x_i w_j = q^{2\delta_{ij}} w_j x_i, \ \ x_i y_j = q^{2\delta_{ij}} y_j x_i
\ee

Furthermore, in every place where building block $B(i)$ is, slightly
different building block $\Bp{i}$ could be used
\be
\Bp{i} = \me_{1/q} \lb \alpha_i q^{H_{[i]}} \hT_{-[i]} \rb q^{\beta_i H_{[i]}} \me_q \lb \gamma_i \hT_{+[i]} q^{-H_{[i]}} \rb   = 
a_i^{H_{[i]}} \me_{1/q} \lb q^{H_{[i]}} \hT_{-[i]} \rb b_i^{H_{[i]}} \me_q \lb  \hT_{+[i]} q^{-H_{[i]}} \rb c_i^{H_{[i]}},
\ee
and {\bf quantum} substitution of variables, which relates newly obtained $g$' to the old one
is of the form of a mutation.
\be
a_i = w_i(1 + q x_i), \ \ c_i = y_i(1 + q x_i), \ \ b_i = \frac{1}{x_i},
\ee 
assuming mutation occured in the building block $B(i)$.

We interpret our results as follows: quantum group $SL(n)_q$ is a cluster  variety,
with very simple cluster data:
\begin{enumerate}
\item $I$ - it is in fact hard to describe what seed and frozen seed look like
\item $I_0$ - since it is being done on like two pages in Fock Goncharov 0508408.
\item $\epsilon_{ij} = offdiag \lb \delta_{ij}, -\delta_{ij} \rb$
\item $d_i \equiv 1$ for all $i$
\end{enumerate}

\section {$SL(2)_q$ case}
In this section we detail all parts of the construction in case simplest example of $SL(2)_q$.
Furthermore, unless explicitly stated, we restrict considerations to fundamental representation.
This is done to make this section clear and illustrarive, although all calculations, of course,
remain valid in higher representations.

\subsection{Quantum algebra's commutational relations and comultiplication rules}
In case of $SL(2)_q$ there are only one positive root $T_{+}$, one negative root $T_{-}$
and one Cartan element $H$.

Commutational relations read
\be
q^H \hT_{\pm} = q^{\pm 1} \hT_{\pm} q^H,\ \ \hT_+ \hT_- - \hT_- \hT_+ = \frac{q^{2H} - q^{-2H}}{q - q^{-1}},
\ee
and comultiplication relations are
\be
\Delta(H) = I \otimes H + H \otimes I, \ \ \Delta(\hT_\pm) = q^H \otimes \hT_\pm + \hT_\pm \otimes q^{-H}
\ee

There are two obvious ways to twist generators, associated to roots, which play crucial role in
the construction below.

One is to consider
\be
\label{twist-positively}
T_+ = q^H \hT_+,\ \And1 \ T_- = \hT_- q^{-H}
\ee
with comultiplication rules
\be
\label{comultiply-positively}
\Delta(T_+) = q^{2H} \otimes T_+ + T_+ \otimes I,\ \ \Delta(T_-) = I \otimes T_- + T_- \otimes q^{-2H},
\ee
and another is
\be
\label{twist-negatively}
T_- = q^H \hT_-,\ \And1 \ T_+ = \hT_+ q^{-H},
\ee
with comultiplication rules
\be
\label{comultiply-negatively}
\Delta(T_-) = q^{2H} \otimes T_- + T_- \otimes I,\ \ \Delta(T_+) = I \otimes T_+ + T_+ \otimes q^{-2H},
\ee

If we restrict ourselves to only consider root-associated generators twisted this or that way or only untwisted,
then in fundamental representation they can be chosen to be
\be
T_+ = \Eoned{1},\ T_- = \Foned{1},\ H = \matd{\frac{1}{2}}{0}{0}{-\frac{1}{2}}
\ee

\subsection{Ansatz for group element in two equivalent forms}
In this simple case we have only one pair of positive-negative simple roots, hence group
element contains only one building block
\be
\label{quantum-group-element-sl2}
g = B(1),
\ee
where $B$ can be chosen to be in one of two equivalent forms.

The former is inspired by Fock-Goncharov construction and reads
\be
\label{building-block-sl2-fg}
B(1) = & w^H \me_q \lb T_+\rb x^H \me_{1/q} \lb T_- \rb y^H & = \\
= & \Honed{w} \Eoned{1} \Honed{x} \Foned{1} \Honed{y} & = \nonumber \\
= & \matd{w^{1/2}x^{1/2}y^{1/2} + w^{1/2}x^{-1/2}y^{1/2}}{w^{1/2}x^{-1/2}y^{-1/2}}
{w^{-1/2}x^{-1/2}y^{1/2}}{w^{-1/2}x^{-1/2}y^{-1/2}} \nonumber,
\ee

The latter comes from Morozov-Vinet considerations and reads
\be
\label{building-block-sl2-mv}
B(1) = & \me_q \lb \psi T_+ \rb q^{\phi H} \me_{1/q} \lb \chi T_- \rb & = \\
= & \Eoned{\psi} \matd{q^{\phi/2}}{0}{0}{q^{-\phi/2}} \Foned{\chi} = \matd{q^{\phi/2} + \psi q^{-\phi/2}\chi}{\psi q^{-\phi/2}}{q^{-\phi/2}\chi}{q^{-\phi/2}} \nonumber.
\ee

In the above formulae $\me_q(x)$ is $q$-exponential (known also as quantum dilogarithm)
\be
\me_q(x) = \sum_{i = 0}^\infty \frac{x^n}{[n]!} q^{-\frac{1}{2}n(n-1)},
\ee
quantum numbers are defined to be symmetrical
$$[n] = \frac{q^n - q^{-n}}{q - q^{-1}}$$
and we have explicitly used the fact, that in fundamental representation $T_+^2 = T_-^2 = 0$,
so exponential is in fact quite simple and not yet so quantum.

For the reasons, that will become clear in section \ref{comm-relations-dual-algebra-sl2},
$T_+$ and $T_-$ in these formulae should be twisted like in \delabel{twist-positively}
(otherwise there won't be any solution to quantum group defining equation).
Looking a little bit ahead one can say, that the other twist convention \delabel{twist-negatively}
is relevant when we change the order of positive-negative roots in the ansatz for $B(1)$.

If variables $w$, $x$, $y$ and $\psi$, $\chi$, $q^\phi$ were commutative, then we could immediately
conclude (by equating two different forms of $B(1)$ \delabel{building-block-sl2-fg} and \delabel{building-block-sl2-mv}),
 that relation between them is
\begin{equation}
  \boxed {
    \label{mv-fg-relation-sl2}
    q^\phi = w x y, \ \ \psi = w,\ \ \chi = y,
  }
\end{equation}

but in quantum (non-commutative) case, everything is not so straightforward.
We have to assume specific form of commutational relations.

Namely, let's assume, that
\begin{equation}
  \label{comm-assumption-sl2}
  \boxed {
    \psi \chi = \chi \psi,\ q^\phi \psi = q^n \psi q^\phi,\ q^\phi \chi = q^n \chi q^\phi,\ \And1 \
    w y = y w,\ x w = q^n w x, \ x y = q^n y x
  }
\end{equation}

for some parameter $n$.
Then
\be
w x y = y x w,\ \And1 \ \lb w^{1/2} x^{1/2} y^{1/2} \rb \lb w^{1/2} x^{1/2} y^{1/2} \rb = w x y,
\ee
and formula \delabel{mv-fg-relation-sl2} is a valid non-commutative substitution of variables.

%% If we denote by $1$ the positive root and by $\overline{1}$ the negative root,
%% then Fock-Goncharov ansatz for group element, corresponding to the ``word'' $1\overline{1}$
%% reads
%% \be
%% g_{cl} & = \Honed{w} \Eoned{1} \Honed{x} \Foned{1} \Honed{y} = & \\
%% & \matd{w^{1/2}x^{1/2}y^{1/2} + w^{1/2}x^{-1/2}y^{1/2}}{w^{1/2}x^{-1/2}y^{-1/2}}
%% {w^{-1/2}x^{-1/2}y^{1/2}}{w^{-1/2}x^{-1/2}y^{-1/2}},
%% \label{g_classd}
%% \ee
%% where we explicitly kept the order of $w$, $x$ and $y$ in the monomials, since we expect this formula to
%% generalize to quantum case, where they are not commutative.


%% On the quantum side, we consider the analog of Morozov-Vinet parametrization, but with the difference, that positive root
%% is to the left of the cartan, not to the right, and different normalization of cartan element is used
%% \be
%% g_{q} = \Eoned{\psi} \matd{q^{\phi/2}}{0}{0}{q^{-\phi/2}} \Foned{\chi} = \matd{q^{\phi/2} + \psi q^{-\phi/2}\chi}{\psi q^{-\phi/2}}{q^{-\phi/2}\chi}{q^{-\phi/2}} \label{g_quantd}
%% \ee

%% One can immediately see, that if we identify
%% \be
%% \boxed{
%% q^\phi = w x y,\ \psi = w,\ \chi = y \label{quant_class}
%% }
%% \ee

%% then in classical limit where $\phi$, $\psi$ and $\chi$ become mutually commutative,
%% formula for the quantum element reproduces formula for the classical element.

%% What {\bf is} nontrivial is that if we suppose (which will turn out to be the case)
%% that commutation relations of the quantum variables take the form
%% \be
%% \boxed{
%% \psi \chi = \chi \psi,\ q^\phi \psi = q^n \psi q^\phi,\ q^\phi \chi = q^n \chi q^\phi, \label{comm_suggest}
%% }
%% \ee

%% then already on the quantum side $w x y$ = $y x w$, and expression (\ref{g_quantd}) can be put into form (\ref{g_classd}), using formula \delabel{quant_class}.

%% \subsubsection{Different roots order}
%% If we consider the other order of roots - the word $\overline{1}1$,
%% i.e. the elements
%% \be
%% g_{cl} & = \Honed{a} \Foned{1} \Honed{b} \Eoned{1} \Honed{c} \\
%% g_q & = \Foned{\alpha} \matd{q^{\beta/2}}{0}{0}{q^{-\beta/2}} \Eoned{\gamma},
%% \ee

%% then the respective relation between $a, b, c$ and $\alpha, \beta, \gamma$ would be
%% \be
%% \boxed{
%%   q^\beta = a b c,\ \ \alpha = \frac{1}{a},\ \ \gamma = \frac{1}{c} \label{quant_class_other}
%% }
%% \ee

\subsection{Commutational relations on the dual algebra from comultiplication}
\label{comm-relations-dual-algebra-sl2}

In this section we use quantum group defining equation $\dg = \gog$ to determine
commutational relations on the dual algebra, following the lines of \cite{MV1}.
 Namely, we take $g$ to be of the form
\delabel{quantum-group-element-sl2}, where $B(1)$ is taken to be of Morozov-Vinet form
\delabel{building-block-sl2-mv}, substitute it into defining equation and provide
such commutational relations for $\phi$, $\psi$ and $\chi$, that the equation becomes valid.

%% \subsubsection{Twisted root generators}
%% Comultiplication rule for Chevalley root generators was
%% \be
%% \comul{\hat{T}_\pm} = q^H \otimes \hat{T}_\pm + \hat{T}_\pm \otimes q^{-H}
%% \ee

%% Since we've changed the order of roots in the expression of the quantum group element, we must also change the
%% definition of the twisted generators, in order for main equation $\dg = \gog$ to have a solution

%% Namely, we twist generators like this
%% \be
%% T_+ = q^H \hat{T}_+,\ T_- = \hat{T}_- q^{-H},
%% \ee
%% and comultiplication rules for these twisted generators become
%% \be
%% \comul{T_+} = q^H \otimes q^H \lsb q^H \otimes \hat{T}_+ + \hat{T}_+ \otimes q^{-H} \rsb = q^{2H} \otimes T_+ + T_+ \otimes I \\
%% \comul{T_-} = \lsb q^H \otimes \hat{T}_- + \hat{T}_- \otimes q^{-H} \rsb q^{-H} \otimes q^{-H} = I \otimes T_- + T_- \otimes q^{-2H}
%% \ee

\subsubsection{q-Exponent factorisation and comultiplication}
q-Exponent has an interesting property
\be
\me_q(y)\me_q(x) = \me_q(x + y)\ \If \ xy = q^2 yx, \label{fact_q}
\ee
which, for convenience, we will also write in terms of $1/q$ instead of $q$
\be
\me_{1/q}(y)\me_{1/q}(x) = \me_{1/q}(x + y)\ if\ xy = q^{-2} yx, \label{fact_over_q}
\ee

Now comultiplication of the part of the group element corresponding to he positive root factorises as
(using \delabel{fact_q}; recall, that we use twisting convention \delabel{twist-positively}, hence
comultiplication rules \delabel{comultiply-positively} apply)
\be
\label{comul-expt-positive-sl2}
\comul{\me_q \lb \psi T_+ \rb} & = \me_q \lb \psi \comul{T_+}\rb = \me_q \lb \psi \lb q^{2H} \otimes T_+ + T_+ \otimes I \rb \rb & \\
& = \me_q \lb \psi T_+ \otimes I\rb \me_q \lb \psi q^{2H} \otimes T_+\rb
\ee

Analogously, using (\ref{fact_over_q}) we get for the negative root
\be
\label{comul-expt-negative-sl2}
\comul{\me_{1/q} \lb \chi T_- \rb} & = \me_{1/q} \lb \chi \comul{T_-}\rb = \me_{1/q} \lb \chi \lb I \otimes T_- + T_- \otimes q^{-2H} \rb \rb & \\
& = \me_{1/q} \lb \chi T_- \otimes q^{-2H}\rb \me_{1/q} \lb \chi I \otimes T_-\rb
\ee

\subsubsection{Convenient notation}

When written directly in terms of $\me_q$, $T_\pm$, $H$ and $\otimes$, formulae quickly become involved,
although these elementary 'letters' often enter only in specific combinations.
In situations like this, commonly used technique is to introduce some higher-level language,
in which to express the statements.

Namely, let's define the following

\be
& \Psi = \me_q \lb \psi T_+ \rb,\ \Chi = \me_{1/q} \lb \chi T_- \rb,\ \Phi = q^H \nonumber \\
& \EPsiD = \comul{\me_\psi},\ \EChiD = \comul{\me_\chi},\ \QPhiD = \comul{Q_\phi} \nonumber \\
& \EPsiL = \me_q \lb \psi T_+ \otimes I\rb,\ \EPsiR = \me_q \lb \psi I \otimes T_+ \rb \nonumber \\
\label{macrolanguage-sl2}
& \EPsiLT = \me_q \lb \psi T_+ \otimes q^{-2H}\rb,\ \EPsiRT = \me_q \lb \psi q^{2H} \otimes T_+ \rb \\
& \EChiL = \me_{1/q} \lb \chi T_- \otimes I\rb,\ \EChiR = \me_{1/q} \lb \chi I \otimes T_- \rb \nonumber \\
& \EChiLT = \me_{1/q} \lb \chi T_- \otimes q^{-2H}\rb,\ \EChiRT = \me_{1/q} \lb \chi q^{2H} \otimes T_- \rb \nonumber \\
& \QPhiL = q^H \otimes I,\ \QPhiR = I \otimes q^H \nonumber 
\ee

Mnemonics is as follows:
\begin{itemize}
\item $\Psi$, $\Chi$ or $\Phi$ means, that the expression it denotes depends, respectively,
on $\psi$, $\chi$ or $\phi$.
\item $L$ in the superscript means that in the expression there is a tensor product,
and something non-trivial (Chevalley generator) is in the {\it left} part of it.
\item Similarly, $R$ means, that something non-trivial is on the {\it right} of the tensor product entering the expression.
\item Finally, $t$ in the superscript means that tensor product involved is ``twisted'', that is the trivial piece
of the tensor product is multiplied by $q^H$ to a certain power.
\end{itemize}

\subsubsection{Solution of defining equation}

Using the notation \delabel{macrolanguage-sl2}, comultiplication rules \delabel{comul-expt-positive-sl2} and
\delabel{comul-expt-negative-sl2}, together with obvious comultiplication rule for the exponent of Cartan element,
can be expressed as
\be
\EPsiD = \EPsiL \EPsiRT,\ \EChiD = \EChiLT \EChiR,\ \QPhiD = \QPhiL \QPhiR
\ee

Now, left and right hand side of defining equation is equal, respectively
\be
& \dg = \EPsiD \QPhiD \EChiD = \EPsiL \EPsiRT \QPhiL \QPhiR \EChiLT \EChiR \label{dg_sl_two} & \\
& \gog = \EPsiL \QPhiL \EChiL \EPsiR \QPhiR \EPsiR \label{gog_sl_two} &
\ee

We immediately see that if
\be
& \EPsiRT \QPhiL = \QPhiL \EPsiR,\ \QPhiR \EChiLT = \EChiL \QPhiR \label{sl_two_first_pass} \\
& \EPsiR \EChiL = \EChiL \EPsiR \label{sl_two_second_pass}
,
\ee
then (\ref{dg_sl_two}) coincides with (\ref{gog_sl_two}).

Equation (\ref{sl_two_second_pass}) is satisfied if
\be
\psi \chi = \chi \psi
\ee

First of (\ref{sl_two_first_pass}) is satisfied if
\be
\psi \lb q^{2H} \otimes T_+ \rb \lb q^{\phi H} \otimes I \rb & = \psi \lb q^{\phi H} \otimes I \rb \lb q^{2H} \otimes T_+ \rb = & \nonumber \\
& = \lb q^{\phi H} \otimes I \rb \psi \lb q^{-2H} \otimes I \rb \lb q^{2H} \otimes T_+ \rb = & \lb q^{\phi H} \otimes I \rb \psi \lb I \otimes T_+ \rb,
\ee

which implies
\be
\psi q^{\phi/2} = q^{\phi/2} \psi q^{-1},\ \that\ \is\ q^\phi \psi = q^2 \psi q^\phi
\ee

Similarly, second of (\ref{sl_two_first_pass}) is satisfied if
\be
q^{\phi H} \chi = \chi q^{\phi H} q^{2H} \\ q^{\phi/2} \chi = \chi q^{\phi/2} q \nonumber \\ q^\phi \chi = q^2 \chi q^\phi \nonumber
\ee

Let's write commutational relations just obtained all on one line

\begin{equation}
\label{comm-relations-psichiphi-sl2}
\boxed{
q^\phi \psi = q^2 \psi q^\phi,\ \ q^\phi \chi = q^2 \chi q^\phi,\ \ \psi \chi = \chi \psi
}
\end{equation}

Thus, our {\it ad hoc} suggestion \delabel{comm-assumption-sl2} proved to be true for commutational
relations on $\psi$, $\chi$ and $\phi$, with $n = 2$.
Now \delabel{mv-fg-relation-sl2}, understood as {\it quantum} substitution of variables is sufficient to
prove, that commutational relations for $x$, $y$ and $w$ are also of the form \delabel{comm-assumption-sl2},
also with $n = 2$
\begin{equation}
\label{comm-relations-wxy-sl2}
\boxed{
x w = q^2 w x,\ \ x y = q^2 y x,\ \ w y = y w
}
\end{equation}
Hence, formula \delabel{mv-fg-relation-sl2} is indeed a full non-commutative change of variables on the
dual algebra of the quantum group $SL_q(2)$.

%% \subsection{Heisenbergisation of commutation relations}

%% Let's rewrite commutation relations, say, between, $\psi$ and $q^\phi$ in terms of Planck constant
%% \be
%% e^{\hbar \phi} \psi = e^{2\hbar} \psi e^{\hbar \phi}
%% \ee

%% Note that if we substitute $\phi = \frac{\partial}{\partial x}$ and $\psi = e^{2x}$, then above equality remains valid.
%% This means, we've found representation of $\phi$ and $\psi$.

%% Commutation relation between $\frac{\partial}{\partial x}$ and $2x$ is
%% \be
%% \lsb \partial_x, 2x \rsb = 2,
%% \ee
%% from which we can conclude, that if we introduce new variable $\mu$
%% \be
%% \psi = e^{2 \mu},
%% \ee
%% then we would have the following commutation relation between $\phi$ and $\mu$
%% \be
%% \lsb \phi, \mu \rsb = 1
%% \ee

%% Analogously,
%% \be
%% \chi = e^{2\nu},\ \ \lsb \phi, \nu \rsb = 1
%% \ee

%% If we introduce logarithmic coordinates on the ``classical'' side
%% \be
%% w = e^{\tw},\ y = e^{\ty},\ x = e^{\tx}
%% \ee

%% then our quantum-classical correspondance \delabel{quant_class} can be rewritten as
%% \be
%% e^{\hbar \phi} = e^{\tw} e^{\tx} e^{\ty}\\
%% e^{2\mu} = e^{\tw} \\
%% e^{2\nu} = e^{\ty},
%% \ee
%% from which we derive
%% \be
%% \tw = 2\mu,\ \ty = 2\nu,\ \tx = \ln \lb e^{-2\mu}e^{\hbar \phi} e^{-2\nu}\rb,
%% \ee
%% and expression for $\tx$ simplifies only in classical limit $\hbar \rightarrow 0$, because exponents become mutually commuting and
%% logarithm can be taken, leading
%% \be
%% \tx = \hbar \phi - 2\nu - 2\mu
%% \ee

\subsection{Alternative ansatz for group element}

We can use different form of the building block in the ansatz for the group element.
\be
\label{quantum-group-element-alt-sl2}
g = \Bp{1},
\ee
where building block can be again expressed in two forms
\be
\label{building-block-alt-sl2-fg}
\Bp{1} = & a^H \me_{1/q} \lb T_-\rb b^H \me_{q} \lb T_+ \rb c^H & = \\
= & \Honed{a} \Foned{1} \Honed{b} \Eoned{1} \Honed{c} & = \nonumber \\
= & \matd{a^{1/2}b^{1/2}c^{1/2}}{a^{1/2}b^{1/2}c^{-1/2}}
{a^{-1/2}b^{1/2}c^{1/2}}{a^{-1/2}b^{1/2}c^{-1/2} + a^{-1/2}b^{-1/2}c^{-1/2}} \nonumber,
\ee
or
\be
\label{building-block-alt-sl2-mv}
\Bp{1} = & \me_q \lb \alpha T_- \rb q^{\beta H} \me_{1/q} \lb \gamma T_+ \rb & = \\
= & \Foned{\alpha} \matd{q^{\beta/2}}{0}{0}{q^{-\beta/2}} \Eoned{\gamma}
= \matd{q^{\beta/2}}{q^{\beta/2}\gamma}{\alpha q^{\beta/2}}{\alpha q^{\beta/2}\gamma  + q^{-\beta/2}} \nonumber.
\ee

Note, that although the form of $T_\pm$ in this formulae appears to be the same as in
\delabel{building-block-sl2-fg} and \delabel{building-block-sl2-mv}, here they are twisted like
in \delabel{twist-negatively}. That is, one should be very careful when writing formulae, which include
both types of ansaetze (for example, the ones in Appendix \ref{alpha-beta-gamma-from-chi-psi-phi}, where
we derive commutational relations on $\alpha$, $\beta$ and $\gamma$ from the ones obtained in
\ref{comm-relations-dual-algebra-sl2} for
$\chi$, $\psi$ and $\phi$), as they include additional factors of $q^H$ where appropriate.

Again, naive commutative relation between $(a, b, c)$ and $(\alpha, \beta, \gamma)$ can be
lifted to the full non-commutative relation
\begin{equation}
  \boxed {
    \label{mv-fg-relation-alt-sl2}
    q^\beta = a b c, \ \ \alpha = \frac{1}{a},\ \ \gamma = \frac{1}{c},
  }
\end{equation}
provided the following commutation relations hold
\begin{equation}
  \label{comm-assumption-alt-sl2}
  \boxed {
    \alpha \gamma = \gamma \alpha,\ q^\beta \alpha = q^n \alpha q^\beta,\ q^\beta \gamma = q^n \gamma q^\beta,\ \And1 \
    a c = c a,\ b a = q^{-n} a b, \ b c = q^{-n} c b,
  }
\end{equation}
for some $n$.

\subsubsection{Relation between MV-type ansaetze}
Comparing \delabel{building-block-sl2-mv} to \delabel{building-block-alt-sl2-mv}, one can write
the following four equations
\be
& q^{\phi/2} + \psi q^{-\phi/2} \chi = q^{\beta/2} \\
& \psi q^{-\phi/2} = q^{\beta/2} \gamma \\
& q^{-\phi/2} \chi = \alpha q^{\beta/2} \\
& q^{-\phi/2} = \alpha q^{\beta/2} \gamma + q^{-\beta/2}
\ee

First three are sufficient to express $\alpha$, $\beta$ and $\gamma$ through $\chi$, $\psi$ and $\phi$,
while the fourth provides consistency check.

\be
& q^{\beta/2} = q^{\phi/2} + \psi q^{-\phi/2} \chi \\
& \alpha = q^{-\phi/2} \chi q^{-\phi/2} \lb 1 + \psi q^{-\phi/2} \chi q^{-\phi/2} \rb^{-1}
& \gamma = \lb 1 + q^{-\phi/2} \psi q^{-\phi/2} \chi \rb^{-1} q^{-\phi/2} \psi q^{-\phi/2}
\ee

One can further verify that the form of the commutation relations is preserved by this change of variables, i.e.
\be
q^{\beta} \alpha = q^2 \alpha q^{\beta} \\
q^{\beta} \gamma = q^2 \gamma q^{\beta} \\
\alpha \gamma = \gamma \alpha
\ee

The details of this verification and check, that the fourth equation indeed holds
can be found in Appendix \ref{alpha-beta-gamma-from-chi-psi-phi}

\subsubsection{Relation between FG-type ansaetze}
Comparing \delabel{building-block-sl2-fg} to \delabel{building-block-alt-sl2-fg} and then
solving the obtained equations, one can arrive at another form of relation between
ansaetze, corresponding to different root order.

Expressed like this, it takes particularly simple form
\begin{equation}
\label{quantum-mutation-sl2}
  \boxed{
    a = w (1 + q x),\ \ c = y(1 + q x), \ \ b = \frac{1}{x}
  }
\end{equation}
in which one readily recognizes the quantum cluster transformation (also called the mutation).

Setting $\classlim$, of course, reproduces the classical version of mutation
\be
\label{classical-mutation-sl2}
a = w \lb 1 + x\rb \\
c = y \lb 1 + x\rb \\
b = \frac{1}{x}
\ee

The details of the derivation of \delabel{quantum-mutation-sl2} are presented in Appendix
\ref{derivation-of-quantum-mutation-sl2}.

%% \subsection{Poisson submanifold}
%% Although reduction $\psi = 1$ is not consistent with commutational relations, if we
%% restrict ourselves to only functions, depending on $\chi$ and $q^\phi$, it is still consistent,
%% since we will never get any $\psi$'s while considering such functions, hence will never actually
%% know, that commutation relations did break.

%% Note: in Fock-Goncharov they do precisely that in their example with $sl(2)$ (arXiv:math/0508408),
%% but in classical limit.

\section*{Acknowledgements}

Our work is partly supported by Ministry of Education and Science of
the Russian Federation under contract \?, by RFBR
grants \? by joint grants \?.

\appendix
\section{Appendix. Some explicit calculations not to be placed in the main text}

\section{Commutational relations on $(\alpha\ \beta\ \gamma)$ from that of $(\chi\ \psi\ \phi)$}
\label{alpha-beta-gamma-from-chi-psi-phi}

\section{Mutation formulae}
\subsection{$SL_q(2)$ fundamental representation}
\label{derivation-of-quantum-mutation-sl2}
This derivation is straightforward, yet it is example of calculation,
which already contains some peculiarities, which typically arise when doing non-commutative algebra,
hence we present it here in very detailed form.

We start from the following equations, which come from equating \delabel{building-block-sl2-fg} to \delabel{building-block-alt-sl2-fg}
\be
& w^{1/2}x^{1/2} y^{1/2} + w^{1/2}x^{-1/2}y^{1/2} = a^{1/2}b^{1/2}c^{1/2} \\
& w^{1/2}x^{-1/2}y^{-1/2} = a^{1/2}b^{1/2}c^{-1/2} \\
& w^{-1/2}x^{-1/2}y^{1/2} = a^{-1/2}b^{1/2}c^{1/2} \\
& w^{-1/2}x^{-1/2}y^{-1/2} = a^{-1/2}b^{1/2}c^{-1/2} + a^{-1/2}b^{-1/2}c^{-1/2}
\ee
From second equation we get (multiplying both sides by $c$ to the right)
\be
& w^{1/2}x^{-1/2}y^{-1/2} c = a^{1/2}b^{1/2}c^{1/2},
\ee
which, using the first equation allows us to state that
\be
c = y^{1/2}x^{1/2}w^{-1/2}\lb w^{1/2}x^{1/2}y^{1/2} + w^{1/2}x^{-1/2}y^{1/2} \rb
= y^{1/2} x y^{1/2} + y = y\lb 1 + qx\rb = \lb 1 + \frac{x}{q} \rb y,
\ee
where we used commutational relations \delabel{comm-relations-wxy-sl2}.

Analogously, multiplying both sides of the third equation by a to the left, we get
\be
& a w^{-1/2} x^{-1/2} y^{1/2} = a^{1/2}b^{1/2}c^{1/2},
\ee
which, using the first equation leads
\be
a = \lb w^{1/2}x^{1/2}y^{1/2} + w^{1/2}x^{-1/2}y^{1/2} \rb y^{-1/2}x^{1/2}w^{1/2}
= w^{1/2}x w^{1/2} + w = w \lb 1 + q x \rb
\ee

Now the trickiest part is to calculate the expression for $b$. Naively, one  would want
to multiply the first equation by $a^{-1/2}$ to the left and by $c^{-1/2}$ to the right,
get $b^{1/2}$ and then square it.
However, calculating $a^{1/2}$ is not as simple as taking the product of square roots of both factors
\be
&\sqrt{a} \neq \sqrt{w} \sqrt{1 + q x},
\ee
instead there is this q-Exponential formula \delabel{q-exponential-a-power}, which one will need when
doing calculation in arbitrary representation.
Failure to notice this peculiarity easily leads to incorrect expression for $b$
\be
b \neq \frac{\lb 1 + x\rb^2}{\lb 1 + q x\rb\lb 1 + \frac{x}{q}\rb} \frac{1}{x}.
\ee

Here, on the other hand, we can simply observe, that
\be
\sqrt{a} \sqrt{b} \sqrt{c} \sqrt{a} \sqrt{b} \sqrt{c} = q^{-1/2} a \sqrt{b} \sqrt{c} \sqrt{b} \sqrt{c} = abc,
\ee
and then, using the first equation

\be
b = & a^{-1} \lb \sqrt{a}\sqrt{b}\sqrt{c} \rb^2 c^{-1} & \nonumber \\
= & \lb 1 + qx\rb^{-1}w^{-1}\lsb w^{1/2}x^{1/2}y^{1/2} + w^{1/2}x^{-1/2}y^{1/2} \rsb
\lsb w^{1/2}x^{1/2}y^{1/2} + w^{1/2}x^{-1/2}y^{1/2} \rsb y^{-1}\lb 1 + \frac{x}{q}\rb^{-1}
\\ = & \lb 1 + qx\rb^{-1} \lsb x + \lb q + \frac{1}{q} \rb \frac{1}{x} \rsb \lb 1 + \frac{x}{q}\rb^{-1} = \frac{1}{x}
\nonumber 
\ee

The check, that the fourth equation is valid is trivial, we do not present it here.

\subsection{$SL_q(2)$ first symmetric representation}
\subsection{$SL_q(N)$ fundamental representation}

\begin{thebibliography}{12}

\bibitem{RefName}
 Authors., Journal {\bf issue} (year) pages, arxiv:99999999;

\end{thebibliography}


\end{document}
