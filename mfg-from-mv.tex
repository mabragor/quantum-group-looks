\documentclass{paper}
\usepackage{graphicx}
\usepackage{graphicx,amssymb,amsmath,amsfonts,color,wrapfig}

%\usepackage[cp866]{inputenc}   %  for MiKTeX package
%\usepackage[russian]{babel}

\def\be{\begin{eqnarray}}
\def\ee{\end{eqnarray}}
\def\nn{\nonumber}
\newcommand{\eq}[1]{\begin{equation} #1 \end{equation}}

\def\p{\partial}
\def\tr{{\rm tr}\,}
\def\Tr{{\rm Tr}\,}
\def\tchi{\tilde{\chi}}
\def\tpsi{\tilde{\psi}}
\def\me{\mathcal{E}}
\def\dg{\Delta \lb g \rb}
\def\lb{\left (}
\def\rb{\right )}
\def\lsb{\left [}
\def\rsb{\right ]}
\def\lcb{\left \{}
\def\rcb{\right \}}
\def\lab{\left <}
\def\rab{\right >}
\def\tx{\tilde{x}}
\def\ty{\tilde{y}}
\def\tw{\tilde{w}}

\def\gog{g \otimes g}

\def\classlim{\hbar \rightarrow 0}

%\newcommand{\ll}[1]{\left ( #1 \rb}
\newcommand{\Echi}[1]{\me_{1/q} \lb \chi_#1 T_{-#1} \rb}
\newcommand{\Epsi}[1]{\me_{q} \lb\psi_#1 T_{+#1}\rb}
\newcommand{\Cartan}[1]{\lsb q^{2\phi_i H_i} \rsb}
\newcommand{\EchiFirst}[1]{\me_{1/q} \lb\chi_#1 T_{-#1} \otimes I \rb}
\newcommand{\EchiSecond}[1]{\me_{1/q} \lb\chi_#1 I \otimes T_{-#1} \rb}
\newcommand{\EpsiFirst}[1]{\me_{q} \lb\psi_#1 T_{+#1} \otimes I \rb}
\newcommand{\EpsiSecond}[1]{\me_{q} \lb\psi_#1 I \otimes T_{+#1} \rb}
\newcommand{\CartanFirst}[1]{\lsb q^{2\phi_#1 H_#1} \otimes I \rsb}
\newcommand{\CartanSecond}[1]{\lsb I \otimes q^{2\phi_#1 H_#1}  \rsb}
\newcommand{\CartanCoprod}[1]{\lsb q^{2\phi_#1 H_#1} \otimes q^{2\phi_#1 H_#1}  \rsb}
\newcommand{\EchiPreCoprod}[1]{\me_{1/q} \lb\chi_#1 \Delta \lb T_{-#1} \rb \rb}
\newcommand{\EpsiPreCoprod}[1]{\me_{q} \lb\psi_#1 \Delta \lb T_{+#1} \rb \rb}
\newcommand{\EchiCoprod}[1]{\me_{1/q} \lb\chi_#1 \lb q^{2H_i} \otimes T_{-#1} + T_{-#1} \otimes I\rb \rb}
\newcommand{\EpsiCoprod}[1]{\me_{q} \lb\psi_#1 \lb I \otimes T_{+#1} + T_{+#1} \otimes q^{-2H_i}\rb \rb}
\newcommand{\EchiTwisted}[1]{\me_{1/q} \lb\chi_#1 q^{2H_i} \otimes T_{-#1} \rb}
\newcommand{\EpsiTwisted}[1]{\me_{q} \lb\psi_#1 T_{+#1} \otimes q^{-2H_i} \rb}

\def\Chi{\mathcal{X}}


%% \newcommand{\EChi}[0]{\me_{1/q} \lb \chi_#1 T_{-#1} \rb}
%% \newcommand{\EPsi}[0]{\me_{q} \lb\psi_#1 T_{+#1}\rb}
\newcommand{\EChiL}[0]{\Chi^L}
\newcommand{\EChiR}[0]{\Chi^R}
\newcommand{\EChiLT}[0]{\Chi^{Lt}}
\newcommand{\EChiRT}[0]{\Chi^{Rt}}
\newcommand{\EChiD}[0]{\Chi^\Delta}
\newcommand{\EPsiL}[0]{\Psi^L}
\newcommand{\EPsiR}[0]{\Psi^R}
\newcommand{\EPsiLT}[0]{\Psi^{Lt}}
\newcommand{\EPsiRT}[0]{\Psi^{Rt}}
\newcommand{\EPsiD}[0]{\Psi^\Delta}
\newcommand{\QPhiL}[0]{\Phi^L}
\newcommand{\QPhiR}[0]{\Phi^R}
\newcommand{\QPhiD}[0]{\Phi^\Delta}

\newcommand{\EPsii}[1]{\Psi_#1}
\newcommand{\EChii}[1]{\Chi_#1}
\newcommand{\QPhii}[1]{\Phi_#1}
\newcommand{\EChiLi}[1]{\Chi_#1^L}
\newcommand{\EChiRi}[1]{\Chi_#1^R}
\newcommand{\EChiLTi}[1]{\Chi_#1^{Lt}}
\newcommand{\EChiRTi}[1]{\Chi_#1^{Rt}}
\newcommand{\EChiDi}[1]{\Chi_#1^\Delta}
\newcommand{\EPsiLi}[1]{\Psi_#1^L}
\newcommand{\EPsiRi}[1]{\Psi_#1^R}
\newcommand{\EPsiLTi}[1]{\Psi_#1^{Lt}}
\newcommand{\EPsiRTi}[1]{\Psi_#1^{Rt}}
\newcommand{\EPsiDi}[1]{\Psi_#1^\Delta}
\newcommand{\QPhiLi}[1]{\Phi_#1^L}
\newcommand{\QPhiRi}[1]{\Phi_#1^R}
\newcommand{\QPhiDi}[1]{\Phi_#1^\Delta}
\newcommand{\EBlockLi}[1]{\EPsiLi{#1} \QPhiLi{#1} \EChiLi{#1}}
\newcommand{\EBlockRi}[1]{\EPsiRi{#1} \QPhiRi{#1} \EChiRi{#1}}
\newcommand{\ComultBlock}[1]{\EPsiLi{#1}\EPsiRTi{#1}\QPhiLi{#1}\QPhiRi{#1}\EChiLTi{#1}\EChiRi{#1}}

\newcommand{\matd}[4]{\lb \begin{array}{cc}
#1 & #2 \\ #3 & #4
\end{array} \rb}

\newcommand{\comul}[1]{\Delta \lb #1 \rb}

\newcommand{\delabel}[1]{(\ref{#1})}

\newcommand{\Hone}[1]{\lb \begin{array}{ccc}
#1^{2/3} & 0 & 0 \\ 0 & #1^{-1/3} & 0 \\ 0 & 0 & #1^{-1/3}
\end{array} \rb}
\newcommand{\Honed}[1]{\lb \begin{array}{ccc}
#1^{1/2} & 0 \\ 0 & #1^{-1/2}
\end{array} \rb}
\newcommand{\Eone}[1]{\lb \begin{array}{ccc}
1 & #1 & 0 \\ 0 & 1 & 0 \\ 0 & 0 & 1
\end{array} \rb}
\newcommand{\Eoned}[1]{\lb \begin{array}{cc}
1 & #1 \\ 0 & 1
\end{array} \rb}
\newcommand{\Fone}[1]{\lb \begin{array}{ccc}
1 & 0 & 0 \\ #1 & 1 & 0 \\ 0 & 0 & 1
\end{array} \rb}
\newcommand{\Foned}[1]{\lb \begin{array}{cc}
1 & 0 \\ #1 & 1
\end{array} \rb}


\newcommand{\Htwo}[1]{\lb \begin{array}{ccc}
#1^{1/3} & 0 & 0 \\ 0 & #1^{1/3} & 0 \\ 0 & 0 & #1^{-2/3}
\end{array} \rb}
\newcommand{\Etwo}[1]{\lb \begin{array}{ccc}
1 & 0 & 0 \\ 0 & 1 & #1 \\ 0 & 0 & 1
\end{array} \rb}
\newcommand{\Ftwo}[1]{\lb \begin{array}{ccc}
1 & 0 & 0 \\ 0 & 1 & 0 \\ 0 & #1 & 1
\end{array} \rb}



%\input{head.tex}

%%%%%%%%%%%%%%%%%%%%%%%%%%%%%%%%%%%%%%%%%%%%%%%%%%%%%%%%%%%%%%%%%%%%%%%%
%%%%%%%%%               SPACE FILLING SETTINGS               %%%%%%%%%%%
%%%%%%%%%%%%%%%%%%%%%%%%%%%%%%%%%%%%%%%%%%%%%%%%%%%%%%%%%%%%%%%%%%%%%%%%
\textheight 21.5cm
\textwidth 19.0cm
% \voffset=-1.2in
\voffset=-0.5in
% \voffset= - 1.85in
\hoffset= - 1.5in         % switch off for draft style
%%%%%%%%%%%%%%%%%%%%%%%%%%%%%%%%%%%%%%%%%%%%%%%%%%%%%%%%%%%%%%%%%%%%%%%%

\begin{document}


\thispagestyle{empty}

\baselineskip14pt

\hfill ITEP/TH-???

\bigskip

\bigskip

\centerline{\LARGE{Marshakov-Fock-Goncharov from Morozov-Vinet}}

\centerline{ABSTRACT}

\bigskip

{\footnotesize
The goal of the present note is to derive Marshakov-Fock-Goncharov construction for
classical integrable system from free-field representation of quantum group element of Morozov-Vinet.
$SL(3)$ case is worked out in great detail.
% Hints of the derivation for $SL(n)$ case for generic $n$ are given.
% Conjecture is made for other Lie algebras.
}

It is best to introduce complications one at a time, hence we start with the simplest nontrivial

\section {$SL(2)$ case}

\subsection{Preliminaries}
In the case of $SL(2)$ we have only one positive root, one negative root and one cartan element.
In fundamental representation they are equal, respectively
\be
T_+ = \Eoned{1},\ T_- = \Foned{1},\ H = \matd{\frac{1}{2}}{0}{0}{-\frac{1}{2}}
\ee

\subsection{Sketch of relation between quantum and classical}

If we denote by $1$ the positive root and by $\overline{1}$ the negative root,
then Fock-Goncharov ansatz for group element, corresponding to the ``word'' $1\overline{1}$
reads
\be
g_{cl} & = \Honed{w} \Eoned{1} \Honed{x} \Foned{1} \Honed{y} = & \\
& \matd{w^{1/2}x^{1/2}y^{1/2} + w^{1/2}x^{-1/2}y^{1/2}}{w^{1/2}x^{-1/2}y^{-1/2}}
{w^{-1/2}x^{-1/2}y^{1/2}}{w^{-1/2}x^{-1/2}y^{-1/2}},
\label{g_classd}
\ee
where we explicitly kept the order of $w$, $x$ and $y$ in the monomials, since we expect this formula to
generalize to quantum case, where they are not commutative.


On the quantum side, we consider the analog of Morozov-Vinet parametrization, but with the difference, that positive root
is to the left of the cartan, not to the right, and different normalization of cartan element is used
\be
g_{q} = \Eoned{\psi} \matd{q^{\phi/2}}{0}{0}{q^{-\phi/2}} \Foned{\chi} = \matd{q^{\phi/2} + \psi q^{-\phi/2}\chi}{\psi q^{-\phi/2}}{q^{-\phi/2}\chi}{q^{-\phi/2}} \label{g_quantd}
\ee

One can immediately see, that if we identify
\be
\boxed{
q^\phi = w x y,\ \psi = w,\ \chi = y \label{quant_class}
}
\ee

then in classical limit where $\phi$, $\psi$ and $\chi$ become mutually commutative,
formula for the quantum element reproduces formula for the classical element.

What {\bf is} nontrivial is that if we suppose (which will turn out to be the case)
that commutation relations of the quantum variables take the form
\be
\boxed{
\psi \chi = \chi \psi,\ q^\phi \psi = q^\alpha \psi q^\phi,\ q^\phi \chi = q^\alpha \chi q^\phi, \label{comm_suggest}
}
\ee

then already on the quantum side $w x y$ = $y x w$, and expression (\ref{g_quantd}) can be put into form (\ref{g_classd}), using formula \delabel{quant_class}.

\subsubsection{Different roots order}
If we consider the other order of roots - the word $\overline{1}1$,
i.e. the elements
\be
\be
g_{cl} & = \Honed{a} \Foned{1} \Honed{b} \Eoned{1} \Honed{c} \\
g_q & = \Foned{\alpha} \matd{q^{\beta/2}}{0}{0}{q^{-\beta/2}} \Eoned{\gamma},
\ee

then the respective relation between $a, b, c$ and $\alpha, \beta, \gamma$ would be
\be
\boxed{
  q^\beta = a b c,\ \ \alpha = \frac{1}{a},\ \ \gamma = \frac{1}{c} \label{quant_class_other}
}
\ee

\subsection{Correct quantum commutators from comultiplication}

\subsubsection{Twisted root generators}
Comultiplication rule for Chevalley root generators was
\be
\comul{\hat{T}_\pm} = q^H \otimes \hat{T}_\pm + \hat{T}_\pm \otimes q^{-H}
\ee

Since we've changed the order of roots in the expression of the quantum group element, we must also change the
definition of the twisted generators, in order for main equation $\dg = \gog$ to have a solution

Namely, we twist generators like this
\be
T_+ = q^H \hat{T}_+,\ T_- = \hat{T}_- q^{-H},
\ee
and comultiplication rules for these twisted generators become
\be
\comul{T_+} = q^H \otimes q^H \lsb q^H \otimes \hat{T}_+ + \hat{T}_+ \otimes q^{-H} \rsb = q^{2H} \otimes T_+ + T_+ \otimes I \\
\comul{T_-} = \lsb q^H \otimes \hat{T}_- + \hat{T}_- \otimes q^{-H} \rsb q^{-H} \otimes q^{-H} = I \otimes T_- + T_- \otimes q^{-2H}
\ee

\subsubsection{q-Exponent factorisation and macrolanguage}
q-Exponent has an interesting property
\be
\me_q(y)\me_q(x) = \me_q(x + y)\ if\ xy = q^2 yx, \label{fact_q}
\ee
which, for convenience, we will also write in terms of $1/q$ instead of $q$
\be
\me_{1/q}(y)\me_{1/q}(x) = \me_{1/q}(x + y)\ if\ xy = q^{-2} yx, \label{fact_over_q}
\ee

Now comultiplication of the part of the group element corresponding to he positive root factorises as (using (\ref{fact_q}))
\be
\comul{\me_q \lb \psi T_+ \rb} & = \me_q \lb \psi \comul{T_+}\rb = \me_q \lb \psi \lb q^{2H} \otimes T_+ + T_+ \otimes I \rb \rb & \\
& = \me_q \lb \psi T_+ \otimes I\rb \me_q \lb \psi q^{2H} \otimes T_+\rb
\ee

Analogously, using (\ref{fact_over_q}) we get for the negative root
\be
\comul{\me_{1/q} \lb \chi T_- \rb} & = \me_{1/q} \lb \chi \comul{T_-}\rb = \me_{1/q} \lb \chi \lb I \otimes T_- + T_- \otimes q^{-2H} \rb \rb & \\
& = \me_{1/q} \lb \chi T_- \otimes q^{-2H}\rb \me_{1/q} \lb \chi I \otimes T_-\rb
\ee

At this point it is convenient to introduce some notations, which will be like macrolanguage
\be
\Psi \equiv \me_q \lb \psi T_+ \rb,\ \Chi \equiv \me_{1/q} \lb \chi T_- \rb,\ \Phi = q^H \\
\EPsiD \equiv \comul{\me_\psi},\ \EChiD \equiv \comul{\me_\chi},\ \QPhiD \equiv \comul{Q_\phi} \\
\EPsiL = \me_q \lb \psi T_+ \otimes I\rb,\ \EPsiR = \me_q \lb \psi I \otimes T_+ \rb \\
\EPsiLT = \me_q \lb \psi T_+ \otimes q^{-2H}\rb,\ \EPsiRT = \me_q \lb \psi q^{2H} \otimes T_+ \rb \\
\EChiL = \me_{1/q} \lb \chi T_- \otimes I\rb,\ \EChiR = \me_{1/q} \lb \chi I \otimes T_- \rb \\
\EChiLT = \me_{1/q} \lb \chi T_- \otimes q^{-2H}\rb,\ \EChiRT = \me_{1/q} \lb \chi q^{2H} \otimes T_- \rb \\
\QPhiL = q^H \otimes I,\ \QPhiR = I \otimes q^H
\ee

Using this notation, rules for comultiplication can be expressed as
\be
\EPsiD = \EPsiL \EPsiRT,\ \EChiD = \EChiLT \EChiR,\ \QPhiD = \QPhiL \QPhiR
\ee

Now we are ready to tackle derivation of commutation relations for $\phi$, $\psi$ and $\chi$.
\be
\gog = \EPsiL \QPhiL \EChiL \EPsiR \QPhiR \EPsiR \label{gog_sl_two} \\ 
\dg = \EPsiD \QPhiD \EChiD = \EPsiL \EPsiRT \QPhiL \QPhiR \EChiLT \EChiR \label{dg_sl_two}
\ee

One can see that if
\be
\EPsiRT \QPhiL = \QPhiL \EPsiR,\ \QPhiR \EChiLT = \EChiL \QPhiR \label{sl_two_first_pass} \\
\EPsiR \EChiL = \EChiL \EPsiR \label{sl_two_second_pass}
,
\ee
then (\ref{dg_sl_two}) coincides with (\ref{gog_sl_two}).

Equation (\ref{sl_two_second_pass}) is satisfied if
\be
\psi \chi = \chi \psi
\ee

First of (\ref{sl_two_first_pass}) is satisfied if
\be
\psi \lb q^{2H} \otimes T_+ \rb \lb q^{\phi H} \otimes I \rb = \psi \lb q^{\phi H} \otimes I \rb \lb q^{2H} \otimes T_+ \rb =
\lb q^{\phi H} \otimes I \rb \psi \lb q^{-2H} \otimes I \rb \lb q^{2H} \otimes T_+ \rb = \lb q^{\phi H} \otimes I \rb \psi \lb I \otimes T_+ \rb,
\ee

which implies
\be
\psi q^{\phi/2} = q^{\phi/2} \psi q^{-1},\ that\ is\ q^\phi \psi = q^2 \psi q^\phi
\ee

Similarly, second of (\ref{sl_two_first_pass}) is satisfied if
\be
q^{\phi H} \chi = \chi q^{\phi H} q^{2H} \\ q^{\phi/2} \chi = \chi q^{\phi/2} q \nonumber \\ q^\phi \chi = q^2 \chi q^\phi \nonumber
\ee

Again, let us write all the commutational relations obtained

\be
\boxed{
q^\phi \psi = q^2 \psi q^\phi,\ \ q^\phi \chi = q^2 \chi q^\phi,\ \ \psi \chi = \chi \psi
}
\ee



Note, that our {\it ad hoc} suggestion \delabel{comm_suggest} on the form of commutation relations proved to be true.
Hence, relation between quantum and ``classical'' variables \delabel{quant_class} is valid already in the quantum case,
which justifies quotes around ``classical''.

%% \subsection{Heisenbergisation of commutation relations}

%% Let's rewrite commutation relations, say, between, $\psi$ and $q^\phi$ in terms of Planck constant
%% \be
%% e^{\hbar \phi} \psi = e^{2\hbar} \psi e^{\hbar \phi}
%% \ee

%% Note that if we substitute $\phi = \frac{\partial}{\partial x}$ and $\psi = e^{2x}$, then above equality remains valid.
%% This means, we've found representation of $\phi$ and $\psi$.

%% Commutation relation between $\frac{\partial}{\partial x}$ and $2x$ is
%% \be
%% \lsb \partial_x, 2x \rsb = 2,
%% \ee
%% from which we can conclude, that if we introduce new variable $\mu$
%% \be
%% \psi = e^{2 \mu},
%% \ee
%% then we would have the following commutation relation between $\phi$ and $\mu$
%% \be
%% \lsb \phi, \mu \rsb = 1
%% \ee

%% Analogously,
%% \be
%% \chi = e^{2\nu},\ \ \lsb \phi, \nu \rsb = 1
%% \ee

%% If we introduce logarithmic coordinates on the ``classical'' side
%% \be
%% w = e^{\tw},\ y = e^{\ty},\ x = e^{\tx}
%% \ee

%% then our quantum-classical correspondance \delabel{quant_class} can be rewritten as
%% \be
%% e^{\hbar \phi} = e^{\tw} e^{\tx} e^{\ty}\\
%% e^{2\mu} = e^{\tw} \\
%% e^{2\nu} = e^{\ty},
%% \ee
%% from which we derive
%% \be
%% \tw = 2\mu,\ \ty = 2\nu,\ \tx = \ln \lb e^{-2\mu}e^{\hbar \phi} e^{-2\nu}\rb,
%% \ee
%% and expression for $\tx$ simplifies only in classical limit $\hbar \rightarrow 0$, because exponents become mutually commuting and
%% logarithm can be taken, leading
%% \be
%% \tx = \hbar \phi - 2\nu - 2\mu
%% \ee

\subsection{Poisson submanifold}
Although reduction $\psi = 1$ is not consistent with commutational relations, if we
restrict ourselves to only functions, depending on $\chi$ and $q^\phi$, it is still consistent,
since we will never get any $\psi$'s while considering such functions, hence will never actually
know, that commutation relations did break.

Note: in Fock-Goncharov they do precisely that in their example with $sl(2)$ (arXiv:math/0508408),
but in classical limit.

\subsection{Mutation}

One can consider different parametrization of group element
\be
g = \Foned{\alpha} \matd{q^{\beta/2}}{0}{0}{q^{-\beta/2}} \Eoned{\gamma} = \matd{q^{\beta/2} }{q^{\beta/2}\gamma}{\alpha q^{\beta/2}}{q^{-\beta/2} + \alpha q^{\beta/2}\gamma} \label{g_quantd_2}
\ee

Thus, transition is given by equations
\be
q^{\phi/2} + \psi q^{-\phi/2} \chi = q^{\beta/2} \\
\psi q^{-\phi/2} = q^{\beta/2} \gamma \\
q^{-\phi/2} \chi = \alpha q^{\beta/2} \\
q^{-\phi/2} = \alpha q^{\beta/2} \gamma + q^{-\beta/2}
\ee

One can explicitly solve these equations w.r.t $\alpha$, $\beta$ and $\gamma$
(fourth equation provides a consistency check) to get
\be
q^{\beta/2} = q^{\phi/2} + \psi q^{-\phi/2} \chi \\
\alpha = q^{-\phi/2} \chi q^{-\phi/2} \lb 1 + \psi q^{-\phi/2} \chi q^{-\phi/2} \rb^{-1}
\gamma = \lb 1 + q^{-\phi/2} \psi q^{-\phi/2} \chi \rb^{-1} q^{-\phi/2} \psi q^{-\phi/2}
\ee

One can further verify that the form of the commutation relations is preserved by this transformation, i.e.
\be
q^{\beta} \alpha = q^2 \alpha q^{\beta} \\
q^{\beta} \gamma = q^2 \gamma q^{\beta} \\
\alpha \gamma = \gamma \alpha
\ee

These formulas look much nicer, when expressed as relation between $x, y, z$ and $a, b, c$
variables, namely (see formulas \delabel{quant_class} and \delabel{quant_class_other})

\be
\boxed{
  a = w (1 + q x),\ \ c = y(1 + q x), \ \ b = \frac{\lb 1 + x\rb^2}{\lb 1 + q x\rb\lb 1 + \frac{x}{q}\rb} \frac{1}{x},
}
\ee

and they reproduce classical formulas for mutation in the limit $\classlim$
\be
a = w \lb 1 + x\rb \\
c = y \lb 1 + x\rb \\
b = \frac{1}{x}
\ee


%% \section{$A(n)$ for $n > 1$}

%% The main obstacle now (since we understand how to restrict consideration to $2g$ leafs) is
%% to solve $\dg = \gog$ equation.

%% It is relatively easy, if ansatz, taken for a group element, is
%% \be
%% g = \prod_i \me_q \lb \psi_i T_{+i}\rb q^{2 \phi_i H_i} \me_{1/q} \lb \chi_i T_{-i}\rb,
%% \ee

%% with $H_i$'s with $1/2$ $-1/2$ somewhere on the diagonal.

%% But I was unable to do it even for $SL(3)$ with ``upper'' Cartan elements $\tilde{H}_i$, which in case
%% of $SL(3)$ are $\tilde{H}_1 = diag \lb 2/3, -1/3, -1/3 \rb$ and $\tilde{H}_2 = diag \lb 1/3, 1/3, -2/3 \rb$,
%% which are the ones, that are present in the MFG construction.

%% The problem is that comultiplication still includes ``untilded'' Cartans.
%% If I would be able to get tilded Cartans into rules of comultiplication of $T_{\pm i}$,
%% \be
%% \Delta{T_{\pm i}} = q^{\tilde{H}_i} \otimes T_{\pm i} + T_{\pm i} \otimes q^{-\tilde{H}_i}
%% \ee
%% or something like that, everything
%% would solve, and will (almost obviously) give in the limit the MFG.

%% But I don't know, whether writing such comultiplication will break something else.

%% \subsubsection{$SL(3)$}
%% \be
%% \gog = \EPsiLi{1} \QPhiLi{1} \EChiLi{1} \EPsiLi{2} \QPhiLi{2} \EChiLi{2} \times \EPsiRi{1} \QPhiRi{1} \EChiRi{1} \EPsiRi{2} \QPhiRi{2} \EChiRi{2} \\
%% \dg = \EPsiLi{1} \EPsiRTi{1} \QPhiLi{1} \QPhiRi{1} \EChiLTi{1} \EChiRi{1} \times \EPsiLi{2} \EPsiRTi{2} \QPhiLi{2} \QPhiRi{2} \EChiLTi{2} \EChiRi{2}
%% \ee

%% Take a look at $\EChiLTi{1}$ in the second equation. It clearly must be pushed to the left through
%% $\QPhiRi{1}$ and $\EPsiRTi{1}$. Since $H$'s in $\QPhiRi{1}$ and in $\EChiLTi{1}$ are different,
%% exchange of these will not be able to completely ``neutralize'' the twist of $\EChiLTi{1}$.
%% However, since $\psi$ and $\chi$ are just numbers themselves, not matrices, 
%% exchanging the  $\EPsiRTi{1}$ and $\EChiLTi{1}$ will also not be able to completely neutralize
%% the residual twist of $\EChiLTi{1}$.

%% \subsection{Relation between quantum and classical revised: skewed quasiclassical limit}

%% \subsubsection{Naive $\classlim$}

%% If we just take $\classlim$, then we will get
%% \be
%% \tw = 2\mu,\ \ty = 2\nu,\ \tx = -2\mu - 2\nu, \label{class_corr}
%% \ee
%% which already does not reproduce the expression for $g$ of MFG.

%% Since on the quantum side
%% \be
%% \lsb \tx, \tw \rsb = 2\hbar,\ \lsb \tx, \ty \rsb = 2\hbar,\ \lsb \tw, \ty \rsb = 0,
%% \ee

%% Poisson bracket equals
%% \be
%% \lcb \tx, \tw \rsb = 2,\ \lcb \tx, \ty \rsb = 2, \lcb \tw, \ty \rsb = 0,
%% \ee
%% which, when we substitute \delabel{class_corr} there, simultaneously imply
%% \be
%% \lcb \mu, \nu \rcb = 0,\ and \lcb \mu, \nu \rcb = 1,
%% \ee

%% hence this naive way of taking the classical limit fails.

%% \subsubsection{Skewed $\classlim$}

%% Let's perform the following asymmetric (breaking symmetry between $\mu$ and $\nu$) rescaling of variables 
%% \be
%% \mu = \hbar M \\
%% \nu = N \\
%% \phi = \frac{1}{\hbar} \Phi,
%% \ee

%% then commutation relations between these rescaled variables take the form
%% \be
%% \lsb \Phi, M \rsb = 1 \\
%% \lsb \Phi, N \rsb = \hbar,
%% \ee
%% which means that we still can not put $M$ to zero on the quantum side.


%% However, taking the limit $\hbar \rightarrow 0$ now results in
%% \be
%% \tw = \lim_{\hbar \rightarrow 0} 2 \hbar M = 0,\ \ty = 2N,\ \tx = \Phi - 2N, \label{class_corr_skewed},
%% \ee

%% which imply that $g$ becomes of MFG form ($w$ added for symmetry later had zeroed out).

%% Poisson brackets of $\tx$ $\ty$ and $\tw$ now impose the following equations
%% \be
%% \lcb \Phi - 2N, M\rcb = 2 \\
%% \lcb M, N \rcb = 0 \\
%% \lcb \Phi - 2N, 2N\rcb = 2,
%% \ee
%% which are resolved as
%% \be
%% \lcb M, N\rcb = 0, \lcb \Phi, M \rcb = 1, \lcb \Phi, N \rcb = 1
%% \ee

%% Note, that field $M$ had dissappeared from expression of the group element $g$ in the limit.
%% This means, that it is not contained in the integrals of motion, generated by it, which generate the dynamics.
%% This, in turn, means, that we can put $M$ to any value, for instance, zero and consider
%% motion in only $\Phi$ and $N$ variables.

%% Note that Poisson bracket of $\tx$ and $\ty$ equals $2$, which also coincides with what we have in MFG for SL(2).

%% \subsection{Morale}

%% So, as we see, classical MFG system is a peculiar limit of a quantum system, where some
%% degrees of freedom are not properly fixed, but rather decouple from the rest of the system.
%% If we recall, that system on the classical side is in fact Toda chain, this seems natural,
%% because Toda chain itself is a peculiar Inozemtsev limit of more natural Calogero system.


%% \section{The case of $SL(3)$}

%% \subsection{Relation between variables}

%% Since there are actually many variants of Morozov-Vinet construction, we need to guess the relevant one.

%% Consider (slightly extended, for symmetricity reasons) building block of MFG

%% \be
%% B_{1 cl} \equiv & H_1(w_1) E_1 H_1(x_1) F_1 H_1(y_1) & = \\
%% & \Hone{w_1} \Eone{1} \Hone{x_1} \Fone{1} \Hone{y_1} & = \\
%% = & \lb \begin{array}{ccc}
%% w_1^{2/3} x_1^{2/3} y_1^{2/3} + w_1^{2/3} x_1^{-1/3} y_1^{-1/3} & w_1^{2/3} x_1^{-1/3} y_1^{-1/3} & 0 \\
%%  w_1^{-1/3} x_1^{-1/3} y_1^{2/3} & w_1^{-1/3} x_1^{-1/3} y_1^{-1/3} & 0 \\
%%  0 & 0 & w_1^{-1/3} x_1^{-1/3} y_1^{-1/3}
%% \end{array} \rb & 

%% \ee

%% Hence, corresponding building block on the quantum side should be
%% \be
%% B_{1 q} = & \Eone{\psi_1} \lb \begin{array}{ccc}
%% q^{2/3 \phi_1} & 0 & 0 \\ 0 & q^{-1/3 \phi_1} & 0 \\ 0 & 0 & q^{-1/3 \phi_1}
%% \end{array} \rb \Fone{\chi_1} & = \\
%% = & \lb \begin{array}{ccc}
%% q^{2/3 \phi_1} + \psi_1 q^{-1/3 \phi_1} \chi_1 & \psi_1 q^{-1/3 \phi_1} & 0 \\
%% q^{-1/3 \phi_1} \chi_1 & q^{-1/3 \phi_1} & 0 \\
%%  0 & 0 & q^{-1/3 \phi_1}
%% \end{array} \rb
%% \ee

%% Comparing $B_{1q}$ to $B_{1cl}$ we get that quantum and classical variables are related as
%% \be
%% q^{\phi_1} = w_1 x_1 y_1 \\
%% \psi_1 = w_1 \\
%% \chi_1 = y_1
%% \ee

%% Considering building blocks related to the second root:
%% \be
%% B_{2 cl} \equiv & H_2(w_2) E_2 H_2(x_2) F_2 H_2(y_2) & = \\
%% & \Htwo{w_2} \Etwo{1} \Htwo{x_2} \Ftwo{1} \Htwo{y_2} & = \\
%% = & \lb \begin{array}{ccc}
%% w_1^{2/3} x_1^{2/3} y_1^{2/3} + w_1^{2/3} x_1^{-1/3} y_1^{-1/3} & w_1^{2/3} x_1^{-1/3} y_1^{-1/3} & 0 \\
%%  w_1^{-1/3} x_1^{-1/3} y_1^{2/3} & w_1^{-1/3} x_1^{-1/3} y_1^{-1/3} & 0 \\
%%  0 & 0 & w_1^{-1/3} x_1^{-1/3} y_1^{-1/3} 
%% \end{array} \rb
%% \ee

%% \be
%% B_{2 q} = & \Etwo{\psi_2} \lb \begin{array}{ccc}
%% q^{1/3 \phi_2} & 0 & 0 \\ 0 & q^{1/3 \phi_2} & 0 \\ 0 & 0 & q^{-2/3 \phi_2}
%% \end{array} \rb \Ftwo{\chi_2} & = \\
%% = & \lb \begin{array}{ccc}
%% q^{1/3 \phi_2} & 0 & 0 \\
%% 0 & q^{1/3 \phi_2} + \psi_2 q^{-2/3 \phi_2} \chi_2 & \psi_2 q^{-2/3 \phi_2} \\
%%  0 & q^{-2/3 \phi_2} \chi_2 & q^{-2/3 \phi_2}
%% \end{array} \rb
%% \ee


%% gives analogous relations
%% \be
%% q^{\phi_2} = w_2 x_2 y_2 \\
%% \psi_2 = w_2 \\
%% \chi_2 = y_2
%% \ee

%% \subsection{Commutation relations on the quantum side}

\section{$SL(3)$ case}

\subsection{Commutational relations}
\be
g  = \Psi_1 \Phi_1 \Chi_1 \Psi_2 \Phi_2 \Chi_2 \\
\gog  = \EBlockLi{1}\EBlockLi{2}\EBlockRi{1}\EBlockRi{2} \\
\dg  = \ComultBlock{1}\ComultBlock{2}
\ee

Commutational relations between variables with index $1$,
as well as commutational relations between variables with index $2$,
are obtained analogously to the $SL(2)$ case.

This allows us to express $\dg$ as
\be
\dg = \EBlockLi{1} \EBlockRi{1} \EBlockLi{2} \EBlockRi{2}
\ee

Now, if all variables with {\it distinct} incides mutually commute,
then $\dg = \gog$.

\section{$SL(n)$ case}


\subsection{Gauss parametrization vs. multi-cartan parametrization}

All matrices ($E_i$, $F_i$ and $H_i$), corresponding to the $i-th$ root contain
only $2\times2$ non-trivial block. Thus, calculations from $SL(2)$ apply.

For the word of roots $1\overline{1}2\overline{2} \dots n \overline{n}$
\be
q^{\phi_i} = w_i x_i y_i,\ \ \psi_i = w_i,\ \ \chi_i = y_i
\ee

For the word of roots $\overline{1}1\overline{2}2 \dots \overline{n} n$
\be
q^{\beta_i} = a_i b_i c_i,\ \ \alpha_i = \frac{1}{a_i},\ \ \gamma_i = \frac{1}{c_i}
\ee

\subsection{Commutational relations}
\be
q^{\phi_i} \psi_j = q^{2\delta_{ij}} \psi_j q^{\phi_i} \\
q^{\phi_i} \chi_j = q^{2\delta_{ij}} \chi_j q^{\phi_i} \\
\chi_i \psi_j = \psi_j \chi_i
\ee

\be
q^{\beta_i} \alpha_j = q^{2\delta_{ij}} \alpha_j q^{\beta_i} \\
q^{\beta_i} \gamma_j = q^{2\delta_{ij}} \gamma_j q^{\beta_i} \\
\alpha_i \gamma_j = \gamma_j \alpha_i
\ee

Thus, Poisson brackets are
\be
\lcb x_i w_j\rcb = 2 \delta_{ij} \\
\lcb x_i y_j\rcb = 2 \delta_{ij} \\
\lcb y_i w_j\rcb = 0
\ee

\be
\lcb b_i a_j\rcb = -2 \delta_{ij} \\
\lcb b_i c_j\rcb = -2 \delta_{ij} \\
\lcb a_i c_j\rcb = 0
\ee

\subsection{Mutations}

\be
  a_i = w_i (1 + q x_i),\ \ c_i = y_i(1 + q x_i), \ \ b_i = \frac{\lb 1 + x_i\rb^2}{\lb 1 + q x_i\rb\lb 1 + \frac{x_i}{q}\rb} \frac{1}{x_i},
\ee

\section{TODO}
\begin{itemize}
\item Lift construction from $3g$ leaf to the whole quantum group
\item What to do with different definitions of ``twisted'' $T_\pm$ generators
for direct $i\overline{i}$ and reversed $\overline{i}i$ order of roots?
\end{itemize}

\end{document}
