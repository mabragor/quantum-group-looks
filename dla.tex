\documentclass{paper}
\usepackage{graphicx}
\usepackage{graphicx,amssymb,amsmath,amsfonts,mathtext,color,wrapfig}

%\usepackage[cp866]{inputenc}   %  for MiKTeX package
%\usepackage[russian]{babel}

\def\be{\begin{eqnarray}}
\def\ee{\end{eqnarray}}
\def\nn{\nonumber}
\newcommand{\eq}[1]{\begin{equation} #1 \end{equation}}

\def\p{\partial}
\def\tr{{\rm tr}\,}
\def\Tr{{\rm Tr}\,}
\def\tchi{\tilde{\chi}}
\def\tpsi{\tilde{\psi}}
\def\me{\mathcal{E}}
\def\dg{\Delta \lb g \rb}
\def\lb{\left (}
\def\rb{\right )}
\def\lsb{\left [}
\def\rsb{\right ]}
\def\lcb{\left \{}
\def\rcb{\right \}}
\def\lab{\left <}
\def\rab{\right >}
\def\tx{\tilde{x}}
\def\ty{\tilde{y}}
\def\tw{\tilde{w}}

\def\gog{g \otimes g}
%\newcommand{\ll}[1]{\left ( #1 \rb}
\newcommand{\Echi}[1]{\me_{1/q} \lb \chi_#1 T_{-#1} \rb}
\newcommand{\Epsi}[1]{\me_{q} \lb\psi_#1 T_{+#1}\rb}
\newcommand{\Cartan}[1]{\lsb q^{2\phi_i H_i} \rsb}
\newcommand{\EchiFirst}[1]{\me_{1/q} \lb\chi_#1 T_{-#1} \otimes I \rb}
\newcommand{\EchiSecond}[1]{\me_{1/q} \lb\chi_#1 I \otimes T_{-#1} \rb}
\newcommand{\EpsiFirst}[1]{\me_{q} \lb\psi_#1 T_{+#1} \otimes I \rb}
\newcommand{\EpsiSecond}[1]{\me_{q} \lb\psi_#1 I \otimes T_{+#1} \rb}
\newcommand{\CartanFirst}[1]{\lsb q^{2\phi_#1 H_#1} \otimes I \rsb}
\newcommand{\CartanSecond}[1]{\lsb I \otimes q^{2\phi_#1 H_#1}  \rsb}
\newcommand{\CartanCoprod}[1]{\lsb q^{2\phi_#1 H_#1} \otimes q^{2\phi_#1 H_#1}  \rsb}
\newcommand{\EchiPreCoprod}[1]{\me_{1/q} \lb\chi_#1 \Delta \lb T_{-#1} \rb \rb}
\newcommand{\EpsiPreCoprod}[1]{\me_{q} \lb\psi_#1 \Delta \lb T_{+#1} \rb \rb}
\newcommand{\EchiCoprod}[1]{\me_{1/q} \lb\chi_#1 \lb q^{2H_i} \otimes T_{-#1} + T_{-#1} \otimes I\rb \rb}
\newcommand{\EpsiCoprod}[1]{\me_{q} \lb\psi_#1 \lb I \otimes T_{+#1} + T_{+#1} \otimes q^{-2H_i}\rb \rb}
\newcommand{\EchiTwisted}[1]{\me_{1/q} \lb\chi_#1 q^{2H_i} \otimes T_{-#1} \rb}
\newcommand{\EpsiTwisted}[1]{\me_{q} \lb\psi_#1 T_{+#1} \otimes q^{-2H_i} \rb}





%\input{head.tex}

%%%%%%%%%%%%%%%%%%%%%%%%%%%%%%%%%%%%%%%%%%%%%%%%%%%%%%%%%%%%%%%%%%%%%%%%
%%%%%%%%%               SPACE FILLING SETTINGS               %%%%%%%%%%%
%%%%%%%%%%%%%%%%%%%%%%%%%%%%%%%%%%%%%%%%%%%%%%%%%%%%%%%%%%%%%%%%%%%%%%%%
\textheight 21.5cm
\textwidth 19.0cm
% \voffset=-1.2in
\voffset=-0.5in
% \voffset= - 1.85in
\hoffset= - 1.5in         % switch off for draft style
%%%%%%%%%%%%%%%%%%%%%%%%%%%%%%%%%%%%%%%%%%%%%%%%%%%%%%%%%%%%%%%%%%%%%%%%

\begin{document}


\thispagestyle{empty}

\baselineskip14pt

\hfill ITEP/TH-???

\bigskip

\bigskip

\centerline{\LARGE{On poissonable sector of quantum groups
}}

\vspace{5ex}

\centerline{\large{\emph{A.Popolitov\footnote{Institute for Theoretical and Experimental Physics, Moscow, Russia; popolit@itep.ru}}}}

\vspace{4ex}

\centerline{ABSTRACT}

\bigskip

{\footnotesize
In \cite{Mars},\cite{FG} there is a certain construction of integrable system being built, based on
classical lie algebra. This construction there seems rther {\it ad hoc}. We show, that this
construction can be obtained from quantum group construction of \cite{MV}, as a peculiar quasiclassical limit,
which involves setting some of quantum coordinates to zero, and simultaneously expanding other coordinates
and deformation parameter $q$ near $1$.
}

\section{Introduction}

???

% so, what is the plan?

\section{Example: sl(3) (A(2))}
\subsection{Classical side}
We have two positive simple roots $e_1$ and $e_2$, two negative simple roots $f_1$ and $f_2$ and
two Cartan elements $h_1$ and $h_2$, which in fundamental represention are given by matrices:
\be
e_1 = \lb \begin{array}{ccc}
  0 & 1 & 0 \\
  0 & 0 & 0 \\
  0 & 0 & 0
\end{array}\rb
\
e_2 = \lb \begin{array}{ccc}
  0 & 0 & 0 \\
  0 & 0 & 1 \\
  0 & 0 & 0
\end{array}\rb \nonumber
\\
f_1 = \lb \begin{array}{ccc}
  0 & 0 & 0 \\
  1 & 0 & 0 \\
  0 & 0 & 0
\end{array}\rb
\
f_2 = \lb \begin{array}{ccc}
  0 & 0 & 0 \\
  0 & 0 & 0 \\
  0 & 1 & 0
\end{array}\rb \nonumber
\\
h_1 = \lb \begin{array}{ccc}
  1 & 0 & 0 \\
  0 & -1 & 0 \\
  0 & 0 & 0
\end{array}\rb
\
h_2 = \lb \begin{array}{ccc}
  0 & 0 & 0 \\
  0 & 1 & 0 \\
  0 & 0 & -1
\end{array}\rb \nonumber
\ee

These matrices are subject to commutational relations:
\be
\lsb h_i e_j \rsb = C_{ij} e_j\ \
\lsb h_i f_j \rsb = -C_{ij} f_j\ \
\lsb e_i f_j \rsb = \delta_{ij} h_j \nonumber,
\ee
where summation on the r.h.s is {\bf not} assumed and $C_{ij}$ is the Cartan matrix
$$
C_{ij} = \lb \begin{array}{cc}
  2 & -1 \\
  -1 & 2
\end{array}\rb
$$
of sl(3).

On top of small (algebra) matrices, we build large (group) ones as follows:
\be
E_1 = e^{e_1} = \lb \begin{array}{ccc}
  1 & 1 & 0 \\
  0 & 1 & 0 \\
  0 & 0 & 1
\end{array}\rb\ \ 
E_2 = e^{e_2} = \lb \begin{array}{ccc}
  1 & 0 & 0 \\
  0 & 1 & 1 \\
  0 & 0 & 1
\end{array}\rb \nonumber \\
F_1 = e^{f_1} = \lb \begin{array}{ccc}
  1 & 0 & 0 \\
  1 & 1 & 0 \\
  0 & 0 & 1
\end{array}\rb\ \ 
F_2 = e^{f_2} = \lb \begin{array}{ccc}
  1 & 0 & 0 \\
  0 & 1 & 0 \\
  0 & 1 & 1
\end{array}\rb \nonumber \\
H_1(x) = e^{h_1 x} = \lb \begin{array}{ccc}
  e^x & 0 & 0 \\
  0 & e^{-x} & 0 \\
  0 & 0 & 1
\end{array}\rb\ \ 
H_2(x) = e^{h_2 x} = \lb \begin{array}{ccc}
  1 & 0 & 0 \\
  0 & e^x & 0 \\
  0 & 0 & e^{-x}
\end{array}\rb \nonumber
\ee

We then consider the following matrix-valued function
$$
g = E_1 H_1(x_1) F_1 H_1(y_1) E_2 H_2(x_2) F_2 H_2(y_2)
$$
and non-trivial coefficients $R^{(1)}$ $R^{(2)}$ of its characteristic polynomial
$$
det(g - \lambda E) = \lambda^3 + R^{(2)}(x, y) \lambda^2 + R^{(1)}(x, y) \lambda + 1
$$

If we than search for Poisson bracket on $x_i$ and $y_i$ in the class of {\it constant}
matrices, such that given this bracket, $R^{(1)}$ and $R^{(2)}$ Poisson-commute,
we find, that the only option is:
$$
\Pi = \lb \begin{array}{cccc}
  0 & 1 & 0 & 0 \\
  -1 & 0 & 0 & 0 \\
  0 & 0 & 0 & 1 \\
  0 & 0 & -1 & 0
\end{array}\rb
$$
in the basis $\lbx_1, y_1, x_2, y_2 \rb$

Thus, we have 2 Poisson-commuting functions on the 4-dimensional phase-space,
i.e. we have an integrable system, which turns out to be (relativistic) Toda chain (orly ???).

Note, however, that this is not exactly the construction of Fock-Goncharov-Marshakov.
Namely, instead of lower-index basis in Lie-algebra, they work in higher-index basis
\be
e^1 \equiv \lb \begin{array}{ccc}
  0 & 1 & 0 \\
  0 & 0 & 0 \\
  0 & 0 & 0
\end{array}\rb
\
e^2 \equiv \lb \begin{array}{ccc}
  0 & 0 & 0 \\
  0 & 0 & 1 \\
  0 & 0 & 0
\end{array}\rb \nonumber
\\
f^1 \equiv \lb \begin{array}{ccc}
  0 & 0 & 0 \\
  1 & 0 & 0 \\
  0 & 0 & 0
\end{array}\rb
\
f^2 \equiv \lb \begin{array}{ccc}
  0 & 0 & 0 \\
  0 & 0 & 0 \\
  0 & 1 & 0
\end{array}\rb \nonumber
\\
h^1 = \lb \begin{array}{ccc}
  \frac{2}{3} & 0 & 0 \\
  0 & -\frac{1}{3} & 0 \\
  0 & 0 & -\frac{1}{3}
\end{array}\rb
\
h^2 = \lb \begin{array}{ccc}
  \frac{1}{3} & 0 & 0 \\
  0 & \frac{1}{3} & 0 \\
  0 & 0 & -\frac{2}{3}
\end{array}\rb, \nonumber
\ee
with commutational relations
$$
\left[  h_i e_j \rsb = \delta_{ij}\ \
\left[  h_i f_j \rsb = -\delta_{ij}\ \
\left[  e_i f_j \rsb = -\delta_{ij}\ \
\left[  e_i f_j \rsb = \delta_{ij} \sum_k C_{jk} h^k
$$

However, all the rest of the construction (map from algebra to group,
form of group element $g$ in terms of $E^i$, $F^i$ and $H^i(x)$,
and consideration of $R^{(i)}$ stays the same.

At the end of the day, (constant) Poisson bracket on $x$ and $y$ turns out to be
$$
\Pi = \lb \begin{array}{cccc}
  0 & 2 & 0 -1 \\
  -2 & 0 & 1 & 0 \\
  0 & -1 & 0 & 2 \\
  1 & 0 & -2 & 0,
\end{array}\rb
$$

which can be encoded in this peculiar picture, which is like fat Dynkin diagram.
However, my personal belief is that all there compications are not necessary and
arise from (slightly) inefficient choice of basis.
In the right basis Poisson bracket is very simple, and does not carry any information
about Lie-algebraic structure - construction becomes sort-of orthogonal to it.

\subsection{Relation between quantum and classical variables in case of A(1)}

Every element of $SL(2)$ can be represented as:
\be
g_{cl} = \lb \begin{array}{cc}
w & 0 \\ 0 & \frac{1}{w}
\end{array} \rb \lb \begin{array}{cc}
1 & 0 \\ 1 & 1
\end{array} \rb \lb \begin{array}{cc}
x & 0 \\ 0 & \frac{1}{x}
\end{array} \rb \lb \begin{array}{cc}
1 & 1 \\ 0 & 1
\end{array} \rb \lb \begin{array}{cc}
y & 0 \\ 0 & \frac{1}{y}
\end{array} \rb = \lb \begin{array}{cc}
wxy & \frac{wx}{y} \\ \frac{xy}{w} & \frac{x}{wy} + \frac{1}{wxy}
\end{array} \rb 
\ee

At the same time, element of $SL_q(2)$ can be represented as:
\be
g_{q} = \lb \begin{array}{cc}
1 & 0 \\ \chi & 1
\end{array} \rb \lb \begin{array}{cc}
q^\phi & 0 \\ 0 & q^{-\phi}
\end{array} \rb \lb \begin{array}{cc}
1 & \psi \\ 0 & 1
\end{array} \rb = \lb \begin{array}{cc}
q^\phi & q^\phi\psi \\ \chi q^\phi & \chi q^\phi \psi + q^{-\phi}
\end{array} \rb,
\ee

Equating, $g_{cl}$ to $g_q$, we get
\be
q^\phi = wxy,\ \ \psi = \frac{1}{y^2},\ \ \chi=\frac{1}{w^2}
\ee

{\bf Note:} it is now unclear, how to take the limit $w \rightarrow 1$ to get original Fock-Goncharov
(since $\chi$ does not commute with the rest of the variables, this is not a valid
reduction on the quantum side).

\subsection{Relation between quantum and classical variables in case of A(n)}

Extrapolating results for $SL(2)$ to the case of $SL(n)$, it is safe to write
\be
q^\phi_i = w_ix_iy_i,\ \ \psi_i = \frac{1}{y_i^2},\ \ \chi_i=\frac{1}{w_i^2},
\ee
where $i$ runs from 1 to $n$.

\subsection{Ansatz for subsector of quantum group}
We take the following ansatz for part of quantum group, which as we expect will contain something, that
will correspond to Fock-Goncharov on the classical side.

\be
g_q = \prod_{i=1}^n \lsb \Echi{i} \Cartan{i} \Epsi{i} \rsb
\ee

{\bf Note:} In case of $SL(3)$ classical element looks like
\be
g_{cl} = & \lb \begin{array}{ccc}
w_1 & 0 & 0 \\ 0 & \frac{1}{w_1} & 0 \\ 0 & 0 & 1
\end{array} \rb \lb \begin{array}{ccc}
1 & 0 & 0 \\ 1 & 1 & 0 \\ 0 & 0 & 1
\end{array} \rb \lb \begin{array}{ccc}
x_1 & 0 & 0 \\ 0 & \frac{1}{x_1} & 0 \\ 0 & 0 & 1
\end{array} \rb \lb \begin{array}{ccc}
1 & 1 & 0 \\ 0 & 1 & 0 \\ 0 & 0 & 1
\end{array} \rb \lb \begin{array}{ccc}
y_1 & 0 & 0 \\ 0 & \frac{1}{y_1} & 0 \\ 0 & 0 & 1
\end{array} \rb \times
& \\ \times &

\lb \begin{array}{ccc}
1 & 0 & 0 \\ 0 & w_2 & 0 \\ 0 & 0 & \frac{1}{w_2}
\end{array} \rb \lb \begin{array}{ccc}
1 & 0 & 0 \\ 0 & 1 & 0 \\ 0 & 1 & 1
\end{array} \rb \lb \begin{array}{ccc}
1 & 0 & 0 \\ 0 & x_2 & 0 \\ 0 & 0 & \frac{1}{x_2}
\end{array} \rb \lb \begin{array}{ccc}
1 & 0 & 0 \\ 0 & 1 & 1 \\ 0 & 0 & 1
\end{array} \rb \lb \begin{array}{ccc}
1 & 0 & 0 \\ 0 & y_2 & 0 \\ 0 & 0 & \frac{1}{y_2}
\end{array} \rb \nonumber,

\ee

that is, we have two more variables $w_1$ and $w_2$, that we should somehow get rid of.

\subsection{Commutational relations on $\chi$, $\psi$ and $\phi$ from $\dg = \gog$}

\subsubsection{$SL(3)$}

\be
\gog = & \EchiFirst{1} \CartanFirst{1} \EpsiFirst{1} \EchiFirst{2} \CartanFirst{2} \EpsiFirst{2} & \label{gtensorproduct} \\
& \EchiSecond{1} \CartanSecond{1} \EpsiSecond{1} \EchiSecond{2} \CartanSecond{2} \EpsiSecond{2} & \nonumber 
\ee

\be
\dg = & \EchiPreCoprod{1} \CartanCoprod{1} \EpsiPreCoprod{1} & \\
& \EchiPreCoprod{2} \CartanCoprod{2} \EpsiPreCoprod{2} & = \nonumber \\
= & \EchiCoprod{1} \CartanCoprod{1} \EpsiCoprod{1} & \\
& \EchiCoprod{2} \CartanCoprod{2} \EpsiCoprod{2} & = \nonumber \\
= & \EchiFirst{1} \EchiTwisted{1} \CartanFirst{1} \CartanSecond{1} \EpsiTwisted{1} \EpsiSecond{1} &  \label{gcoproduct} \\
& \EchiFirst{2} \EchiTwisted{2} \CartanFirst{2} \CartanSecond{2} \EpsiTwisted{2} \EpsiSecond{2}
 \nonumber \ee

Let's gradually transform (\ref{gcoproduct}) to the form (\ref{gtensorproduct}),
figuring out on the fly, what commutational relations are necessary for it.

First, the following equalities

\be
\CartanSecond{1} \EpsiTwisted{1} = \EpsiFirst{1} \CartanSecond{1} \\
\CartanSecond{2} \EpsiTwisted{2} = \EpsiFirst{2} \CartanSecond{2} \\
\EchiTwisted{1} \CartanFirst{1} = \CartanFirst{1} \EchiSecond{1} \\
\EchiTwisted{2} \CartanFirst{2} = \CartanFirst{2} \EchiSecond{2} \\
\EchiSecond{1} \EpsiFirst{1} = \EpsiFirst{1} \EchiSecond{1} \\
\EchiSecond{2} \EpsiFirst{2} = \EpsiFirst{2} \EchiSecond{2} \\
\ee

allow us to bring (\ref{gcoproduct}) to the form
\be
\dg = & \EchiFirst{1}\CartanFirst{1}\EpsiFirst{1} \EchiSecond{1}\CartanSecond{1}\EpsiSecond{1} & \label{ginterm} \\
& \EchiFirst{2}\CartanFirst{2}\EpsiFirst{2} \EchiSecond{2}\CartanSecond{2}\EpsiSecond{2}
\ee

Now to bring (\ref{ginterm}) to (\ref{gtensorproduct}) we must commute
$\EchiSecond{1}\CartanSecond{1}\EpsiSecond{1}$ with $\EchiFirst{2}\CartanFirst{2}\EpsiFirst{2}$.

From above commutational relations for exponents, one can deduce the following
commutational relations for $\chi$ $\psi$ and $\phi$.'
\be
q^\phi_i \psi_j = q^{2 \delta_{ij}} \psi_j q^\phi_i \\
q^\phi_i \chi_j = q^{2 \delta_{ij}} \chi_j q^\phi_i \\
[q^\phi_i, q^\phi_j] = 0 \\
[\chi_i, \psi_j] = 0 \\
\ee

\subsubsection{Commutational relations on $x$ $y$ and $w$.}

Since
\be
x_i = \chi_i^{1/2} q^{\phi_i} \psi_i^{1/2} \\
y_i = \psi_i^{-1/2} \\
w_i = \chi_i^{-1/2},
\ee

we have following non-trivial commutational relations
\be
x_i y_j = q^{- \delta_{ij}} y_j x_i
x_i w_j = q^{- \delta_{ij}} w_j x_i,
\ee

now taking the limit $q = e^\hbar$ $\hbar \rightarrow 0$, we get
\be
\lsb x_i y_j \rsb = - \hbar \delta_{ij} x_i y_j \\
\lsb x_i w_j \rsb = - \hbar \delta_{ij} x_i w_j \\
\ee


If we now recall, that in Fock-Goncharov construction there were another $x$'s, namely
\be
x_i = e^{\tx_i},\ y_i = e^{\ty_i},\ w_i = e^{\tw_i} \\
\tx_i = \ln(x_i),\ \ty_i = \ln(y_i),\ \tw_i = \ln(w_i),
\ee

then for bracket on tilded variables we get
\be
\lsb \tx_i \ty_j \rsb = - \hbar \delta_{ij} \\
\lsb \tx_i \tw_j \rsb = - \hbar \delta_{ij},
\ee

so for $x-y$ sector we get precisely the bracket we got in the naive classical approach
to Fock-Goncharov.

\subsubsection{open questions}
\begin{itemize}
\item how to take the limit $w \rightarrow 1$ ?
\item how to obtain commutational relations with Cartan matrix intertwined there from
quantum group perspective?
\end{itemize}

\subsubsection{$SL(n)$}


\subsection{Quantum side}
On the quantum side there is this construction of Morozov-Vinet.
Arbitrary element of quantum group $SL(3)_q$ is expressed as:
$$
g = e_q^{\chi_1 f_1} e_q^{\chi_2 f_2} e_q^{\chi_3 f_1}
q^{\phi_1 h_1} q^{\phi_2 h_2}
e_q^{\psi_1 e_1} e_q^{\psi_2 e_2} e_q^{\psi_3 e_1},
$$
As you can see, $\psi$ and $\chi$ coordinates are related to {\it simple}
roots, but since there are more coordinates, that simple roots,
correspondance is not one-to-one.

In what follows, we denote this correspondence by square brackets, so,
in particular
$$
[1] = 1\ \ [2] = 2\ \ [3] = 1
$$

$\chi$'s, $\psi$'s and $\phi$'s form Heisenberg-like algebra

\be
\psi_i \chi_j = \chi_j \psi_i \nonumber \\
\psi_i \psi_j = q^{C_{[i][j]} \psi_j \psi_i, i < j \nonumber \\
\chi_i \chi_j = q^{C_{[i][j]} \chi_j \chi_i, i < j \nonumber \\
q^{\phi_i} \psi_j = q^{C_{[i][j]}} \psi_j q^{\phi_i} \nonumber \\
q^{\phi_i} \psi_j = q^{C_{[i][j]}} \psi_j q^{\phi_i} \nonumber
\ee

This product $g$ is an element of quantum group since in satisfies
comultiplication rule ???.

\subsection{Relation}
The striking difference between quantum and classical pictures is,
that while on quantum side coordinates with {\it similar} names
do not commute and with different - commute (i.e $\chi$ commute
with $\psi$ but not (in general) with $\chi$),
on classical side coordinates with {\it distinct} names do not
commute and with similar - commute (i.e $x$ commute with $x$'s, but
not with $y$'s).
Clearly, no naive limit can fix this discrepancy.

However, if one somehow manages to perform change of variables
$$
\chi \rightarrow \tchi
\psi \rightarrow \tpsi
$$
such that $\tchi$'s now commute with $\tchi$'s, but not with
$\tpsi$'s, then, maybe, in these coordinates, classical picture
can be revealed. This turns out to be the case.

One can try to dress fields $\chi$ and $\psi$ with help of either
lower or upper Cartan elements.
Looking a little bit ahead, it is impossible for general $n$ to
disentangle all $\chi$'s from $\chi$'s completely. But, it is always
possible to disentangle $\chi_i$'s with $i = 1 .. n$.
Imposing commutation of only these, primary fields with similar names,
one obtains

(again, I dunno how to write the motivation for general formula in this rather non-specific case)

General formula is:
When dressing with upper Cartans:
$$
\tchi_i = \chi_i e^{-\phi_{[i]}},
$$
and ditto for $\tpsi$.

When dressing with lower Cartans:
$$
\tchi_i = \chi_i e^{-[i]\phi_{[i]}},
$$
and ditto for $\tpsi$.

In this case of $n = 2$ this implies the following commutational
relations of $\tchi$'s with $\tpsi$'s (again, only in primary sector)
$$
\tchi_i \tpsi_j = q^A_{ij} \tpsi_j \tchi_i,
$$
where $A$ is $2 x 2$ matrix  
$$
A = \lb\begin{array}{cc}
0 & -1 \\
1 & 0
\end{array}\rb
$$

We are not concerned with what are the commuational relations with the rest of
$\tchi$'s and $\tpsi$'s, since they are {\it linear} in them, and hence allow the
reduction by setting them all to zero. In following, it is assumed, that
this is done and now we are working with $2 n$ dimensional object.
So, dimension is already as in the classical case, all that's left is
to correctly take the limit.

For that, consider the following
$$
\tchi_i = 1 + \hbar y_i + \hbar^2 z_i + o(\hbar^2)\ \
\tpsi_i = 1 + \hbar x_i + \hbar^2 w_i + o(\hbar^2)\ \
q = 1 + \hbar^2 + o(\hbar^2)
$$

L.h.s of the commutational relation for, say, $\chi_2$ with $\psi_1$ becomes
$$
\tchi_2 \tpsi_1 =
\lb 1 + \hbar y_2 + \hbar^2 z_2 + o(\hbar^2)\rb
\lb 1 + \hbar x_1 + \hbar^2 w_1 + o(\hbar^2)\rb =
1 + \hbar \lb y_2 + x_1 \rb + \hbar^2 \lb y_2 x_1 + z_2 + w_1\rb + o(\hbar^2)
$$
And r.h.s becomes
$$
q \tpsi_1 \tchi_2 = \lb1 + \hbar^2 + o(\hbar^2) \rb
\lb 1 + \hbar x_1 + \hbar^2 w_1 + o(\hbar^2)\rb
\lb 1 + \hbar y_2 + \hbar^2 z_2 + o(\hbar^2)\rb =
1 + \hbar \lb x_1 + y_2 \rb + \hbar^2 \lb x_1 y_2 + 1 w_1 + z_2 \rb + o(\hbar^2)
$$

Zero and first orders in $\hbar$ trivialize, and in second
order in $\hbar$ we have
$$
[x_1 y_2] = -1,
$$
which (up a sign, which is arbitrary), coincides with what we had at the classical side.

\section{General construction}

\subsection{Classical picture}
\subsection{Quantum picture}
\subsection{Relation}

As in the $sl(3)$ case, consideration of ansatz
$$
\tchi_i = \chi_i e^{-\phi_{[i]}},\ \ \tpsi_i = \psi_i e^{-\phi_{[i]}},
$$
or the ansatz
$$
\tchi_i = \chi_i e^{-[i]\phi_{[i]}},\ \ \tpsi_i = \psi_i e^{-[i]\phi_{[i]}},
$$
results in commutation of $\tchi$'s with $\tchi$'s and of $\tpsi$'s with $\tpsi$'s
for $i = 1 .. n$ (for coordinates, directly corresponding to simple roots, for primary coordinates,
for main stratum of coordinates).

And for commutational relations of $\tchi$'s with $\tpsi$'s we have, that they
also commute except when
$$
\chi_i \psi_j = q^{A_{ij}} \psi_j \chi_i,
$$
where
$$
A_{ij} = 0, \except \when j = i + 1,
$$
which, again, already resembles picture from the classical side in {\it lower} Lie algebra
basis.

Taking same kind of quasiclassical limit (simultaneously in $\tchi$ $\t

psi$ and $q$),
completely reproduces then the classical picture.

\end{document}
